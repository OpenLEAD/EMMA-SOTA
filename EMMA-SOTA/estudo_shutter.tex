\subsection{Shutter}
Quando não é possível, em uma passada única, atravessar completamente a
superfície ao realizar o coating, coloca-se uma placa de sacrifício, pois, como mencionado em
\ref{desc_hvof}, não é permitido para ou trocar de direção subre a superfície.

Para evitar a necessidade de sobrepôr placas de sacrifício sobre as pás da
turbina, que podem estar em posição de difícil acesso, foi criado o conceito de
\textit{shutter}. Este consiste em um antepero a ser posicionado entre a chama e
a superfície, porém, diferente da placa de sacrifício, deve ficar preso à
pistola de metalização e ser atuado.

%MATERIAIS


Foram desenvolvidos dois conceitos similares para resolver o problema, o \textit{Shutter
Borboleta} e o \textit{Shutter Padrão}, descritos a seguir.

\subsubsection{Shutter Borboleta}

Esse primeiro conceito é constituido de um disco disposto antre a chama e a
superfície. O disco possui duas aberturas opostas de $90^o$, por onde a chama da
pistola travessa, e as outras duas regiões de $90^o$ compostas pelo material que
servirá de anteparo para a chama a fim de evitar o coating da superfície. 

\subsubsection{Shutter Padrão}