\subsection{Estudo de anteparo de proteção para a chama}
Quando não é possível, em uma passada única, atraves\-sar completamente a
superfície ao realizar o \textit{coating}, coloca-se uma placa de sacrifício, pois, como
menciona\-do em \ref{sec::desc_hvof}, não é permitido para ou trocar de direção
subre a superfície.

Para evitar a necessidade de sobrepôr placas de sacrifício sobre as pás da
turbina, que podem estar em posição de difícil acesso, foi criado o conceito de
\textit{shutter}. Este consiste em um antepero a ser posicionado entre a chama e
a superfície, porém, diferente da placa de sacrifício, deve ficar preso à
pistola de metalização e ser atuado.

A pistola de metalização propaga uma chama de $3000^o$ (\ref{sec::desc_hvof}), assim,
a parte da pistola com maior aquecimento, que é o canhão, é fabricado em
cobre-cromo e é refrigerada por um sistema de circulação de água
gelada. A chapa de sacrifício, por outro lado, fica pouco tempo em contato com
a chama e é constituída de um aço qualquer sem sistema de refrigeração
\ref{sec::desc_hvof}.

Quando se planeja submeter o anteparo à temperatu\-ra extrema de $3000^o$ da
chama, sem um sistema de refrigeração, torna-se necessário o uso de materia\-is
especiais para suportar essa temperatura. Para isso existem materiais
aeroespaciais, conhecidos como Ultra-high-temperature ceramics (UHTCs) ou
cerâmi\-cas de temperaturas ultra-altas, em tradução livre. Esse grupo de
materiais possui diversas sub-famílias com densidades e resistência mecânicas
diferentes. Para os cálculos realizados nessa seção foram consideradas a maior
densidade e menor resistência mecânica entre os compostos das sub-famílias. Ou
seja, foi utilizado aproximadamente \textbf{$15 g/mL$} para a densidade, referente ao
carbeto de tântalo \citep{bansal2005ceramic}. E as características mecânicas
foram referentes ao diboreto de zircônio \citep{diborides}.

Foram desenvolvidos dois conceitos similares para resolver o problema, o \textit{Shutter
Borboleta} e o \textit{Shutter Padrão}, descritos a seguir.

\subsubsection{Shutter Borboleta}

Esse primeiro conceito é constituido de um disco disposto antre a chama e a
superfície. O disco possui duas aberturas opostas de $90^o$, por onde a chama da
pistola travessa, e as outras duas regiões de $90^o$ compostas pelo material que
servirá de anteparo para a chama a fim de evitar o coating da superfície. 

\subsubsection{Shutter Padrão}