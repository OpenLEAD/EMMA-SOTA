\subsection{Acesso pela escotilha inferior}
Essa escotilha localizada no topo do aro câmara possui uma abertura de apenas
35 cm de diâmetro, o que cria um desafio quando se pensa em utiliza-la como
ponto de acesso para um robô. Por outro lado sua proximidade às pás e a área livre fora do aro
câmara criam possibilidades interessantes para seu uso.

\textbf{Vantagens}
\begin{itemize}
  \item Estabilidade da fixação do robô
  \item Ponto de referência, facilitando sistemas de localização, mapeamento, controle e calibração
  \item Pórtico rolante para posicionar o robô na escotilha
\end{itemize}

\textbf{Desvantagens}
\begin{itemize}
  \item Dificuldade de encontrar robôs de tal dimensão (35 cm diâmetro)
  \item Necessidade de retirar e inserir o robô quando rodar as pás
  \item Solução não geral, específica para UHE Jirau
\end{itemize}

A solução mais simples para este acessso é a utilização de um robô industrial
comercial. A escolha do robô está primariamente associada ao seu alcance. Por outro lado, apenas
uma pequena parcela dos robôs comerciais possuem a dimensão necessária para
atravessar a escotilha. Sendo assim, o estudo foi direcionada para o uso do
KUKA Light Weight (LBR iiwa 14 R820), robô cuja diagonal da base é inferior aos
35 cm da escotilha.

O LBR R820 pesa 30 kg, possui 7 eixos e suporta carga de 14kg,
suficiente por uma pequena margem para carregar o equipamento de
revestimento. Entretanto, são necessários estudos aprofundados para valida-lo,
como os requisitos de velocidade e precisão quando percorrendo a trajetória
para a realização do revestimento.

Tendo como objetivo posicionar o LBR R820 em uma posição onde seja capaz de
trabalhar toda a pá, um modelo de base articulada foi proposto. A base
composta de dois elos interligados por uma junta de rotação é fixada na
própria escotilha. Para que seja possível cobrir toda a pá,
a base deve ser capaz de assumir diversas angulações com respeito ao
eixo de inserção, e a junta que conecta os dois segmentos da base também precisa
ter sua posição alterada, o que pode ser realizado manualmente ou com atuador.

Introduzir o robô, composto pelo conjunto base-LBR R820, é uma tarefa cuidadosa
pois a extensão total será maior que a distância do topo do aro câmara ao
cone da turbina. Ou seja, o braço e a base precisarão ser rotacionados
durante o processo de inserção, o que acarretará no desalinhamento do centro de
massa (com relação ao eixo perpendicular à escotilha) e exigirá uma guia para
resistir ao torque gerado por esse desalinhamento.

Com o robô fixado na escotilha e a junta da base travada, são esperados torques
inferiores a 3000 Nm sobre a junta e 4000 Nm sobre a base, durante a operação de
revestimento.

%Ao se avaliar as possibilidades de realizar o revestimento sem a necessidade de
%placas de sacrifício, percebe-se que poderá ser realizado se a junta da
%base estiver paralela à pá (que para efeito de cálculo foi assumida com uma
%angulação de $45^o$ com relação ao eixo da turbina). Para esse caso a junta
% precisa realizar uma rotação à velcidade mínima de 34 rad/min. Essa velocidade possibilita a
%pistola de revestimento percorrer verticalmente a pá sem pausa. Percorrer
%horizontalmente a pá não será possível devido às restrições nas dimensções da
%base. O cálculo foi realizado por um estudo puramente geométrico, utilizando as
%informações reais de extensão da pá e área de trabalho do robô, assumindo um
%modelo simples da pá, sem curvatura.

