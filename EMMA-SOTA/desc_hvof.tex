\subsection{Descrição do processo HVOF}\label{sec::desc_hvof}
O revestimento por aspersão térmica (ou metalização) é um processo em que
partículas aquecidas são pulverizadas em uma superfície a fim de melhorar ou
restaurar suas propriedades e dimensões. O revestimento estende a vida útil do
material, aumentando significantemente a sua resistência à erosão e corrosão.
Os diferentes tipos de metalização são: por chama, arco elétrico, detonação,
chama de alta velocidade (HVOF), plasma, a frio e a quente.

Um sistema de metalização é composto por: uma pistola de aspersão, responsável
pelo derretimento e aceleração das partículas a serem depositadas na
superfície; um alimentador, que fornece o pó (partículas) através de tubos;
um fornecedor do material de combustão; um robô para manipular a pistola; uma
fonte de alimentação elétrica para a pistola; um console de controle para o
sistema.

No caso específico das pás (aço inox 420) das turbinas da usina hidrelétrica de
Jirau, antes da montagem da turbina, a metalização tipo HVOF é realizada em
ambos os lados da pá pela empresa Rijeza com um manipulador industrial de 150 kg
de carga máxima, permitindo controle de vibrações com boa margem de segurança, já que a massa do
sistema pode chegar a 20 kg (cabos e pistola). O tempo
médio do processo é de 6 horas por lado da pá.

Primeiramente, a pá é preparada por jateamento com óxido de alumínio. O HVOF
consiste em alimentar, numa câmara de combustão, o material de revestimento
(carboneto de tungstênio) e uma mistura gasosa do combustível (propano) e
oxigênio. De acordo com os dados fornecidos pela empresa Rijeza, a pistola de 8
Kg projeta uma chama de $3000^oC$, que pulveriza as partículas com velocidade de
700 a 1000 m/s, gerando uma força de recuo de 15 N.

O manipulador robótico deve possuir precisão de 5 mm, a pistola no efetuador
deve permanecer a uma distância que varia entre 230 e 240 mm, e ângulo de $30^0$
a $90^0$, em relação à superfície. O manipulador deve ser capaz de mover a
pistola a velocidade constante de 40 m/min, e não pode estar fixa em uma posição
da pá por muito tempo (parada), pois há acúmulo de material, deformando a
superfície. Trocas de direção ou sentido na movimentação do manipulador são
considerados como parada, logo as trocas deverão ser realizadas em áreas
exteriores à superfície da pá ou chapas de sacrifício são utilizadas. 

Placas de sacrifício, ou mascaramento, são chapas colocadas em regões onde as
peça não podem ser jateadas ou revestidas. Geralmente uma chapa de qualquer tipo
de aço pode ser utilizada, pois a chama não fica parada sobre ela por um longo
período, não aquecendo-a o suficiente para danificar. Quando a pistola fica
parada, em funcionamento, a chama é apontada para algum lugar onde não tenha
obstáculos.

 % As informações do processo
% podem ser observadas na figura~\ref{fig::hvof}.
 
%\begin{figure}[h!]	
%	\includegraphics[width=\columnwidth]{figs/intro/hvof.pdf}
%	\caption{Foto do efetuador do manipulador e pistola HVOF.}
%	\label{fig::hvof}
%\end{figure}

Em relação às condições de operação, o espaço da aplicação HVOF é confinado, com
excesso de ruído de 100 a 140 dB, gases nocivos e com risco de explosão podem
ser exalados, a pá pode atingir temperaturas de até $110^oC$, as condições de
umidade e temperatura devem ser ideais para o processo e há perda de $40\%$
das partículas pulverizadas  \citep{wu2006rebound}, que são espalhadas pelo
ambiente. Portanto, a operação deve ser remota, não há presença de pessoas \textit{in loco}, os gases
presentes e umidade/temperatura devem ser constantemente monitorados, o robô
manipulador é selado e as partículas desperdiçadas devem ser removidas
(limpeza). Como forma de segurança contra gases explosivos, o desligamento do
sistema é imediato por corte de gás, porém, em caso de falta de energia, o
manipulador será desligado, mas a chama não se apagará.

A qualidade do revestimento é geralmente avaliada por um instrumento que
realiza a medida de porosidade, oxidação, dureza e rugosidade da superfície. O
processo é realizado manualmente, de maneira rápida e fácil, por um operador e
pontos específicos da pá são analisados.

A tabela~\ref{tab::hvof} resume as restrições e especificações do
projeto:

\begin{center}
\begin{tabular}{  c | c  }
  \hline
  \textbf{Componente} & \textbf{Dado} \\ \hline
  Material da pá & Aço Inox 420 \\ \hline
  Material do aro câmara & Ferromagnético  \\ \hline
  Massa da pistola HVOF & 8 Kg  \\ \hline
  Massa dos cabos HVOF & 12 Kg  \\ \hline
  Tempo HVOF por pá & 6 horas \\ \hline
  Temperatura da chama HVOF & $3000^oC$ \\ \hline
  Recuo da pistola & 15 N \\ \hline
  Precisão do manipulador& 5 mm \\ \hline
  Distância pistola-pá & 230-240 mm \\ \hline
  Ângulo pistola-pá & $30^o$-$90^o$ \\ \hline
  Velocidade do manipulador & 40 m/s \\ \hline
  Ruído HVOF & 100 a 140 dB \\ \hline
  Temepratura da pá & $110^oC$ \\
  \hline
\end{tabular}
\captionof{table}{Dados principais do processo HVOF}
%\caption{Dados principais do processo de metalização HVOF}
\label{tab::hvof}
\end{center}

%Sistemas robóticos não devem utilizar magnetismo como meio de aderência, já que
%o aço inox 420 não apresenta alta permeabilidade magnética e a alta temperatura
%da pá deve inviabilizar essa solução. Adesão por ventosas é uma solução
%viável, pois material não causa dano ao revestimento, porém a escolha do
%material da ventosa deve ser estudado,já que a pá quente pode ocasionar em
%perda de sucção, como em ventosas emborrachadas.

