\subsection{Descrição do processo HVOF}
%TODO Renan: Descrição do processo HVOF (questoes HVOF) e questoes
% relacionadas
O revestimento por asperção térmica (ou metalização) é um processo em que
partículas aquecidas são pulverizadas em uma superfície a fim de melhorar ou
restaurar suas propriedades e dimensões. O revestimento estende a vida útil do
material, aumentando significantemente a sua resistência à erosão e corrosão.
Os diferentes tipos de metalização são: por chama, arco elétrico, detonação,
chama de alta velocidade (HVOF), plasma, a frio e a quente.

Um sistema de metalização é composto por: uma pistola de aspersão, responsável
pelo derretimento e aceleração das partículas a serem depositadas na
superfície; um alimentador, que fornece o pó  

A typical thermal
spray system consists of the following:

Spray torch (or spray gun) - the core device performing the melting and
acceleration of the particles to be deposited 
Feeder - for supplying the
powder, wire or liquid to the torch through tubes.
Media supply - gases or liquids for the generation of the flame or plasma jet,
gases for carrying the powder, etc.
Robot - for manipulating the torch or the substrates to be coated
Power supply - often standalone for the torch
Control console(s) - either integrated or individual for all of the above

No caso específico das pás das turbinas da usina hidrelétrica de Jirau, a
metalização é realizada por HVOF. Este processo consiste em alimentar, numa
câmara de combustão, o material de revestimento (carboneto de tungstênio) e uma
mistura gasosa do combustível (propano) e oxigênio. onde é inflamada e queimada
continuamente. O gás quente resultante a uma pressão de 1 MPa emana através de um cano onde atinge velocidades de 1800 m/s transportando as partículas até à superfície a
revestir. Ao contrário da Metalização por detonação, este processo funciona em
contínuo.8

Como combustível o HVOF pode usar vários tais como os gases Hidrogénio, Metano,
Acetileno, etc e líquidos Querosene, etc.

O revestimento de metalização por HVOF, pode chegar a 12 mm de espessura e é
usado para depositar materiais resistentes à abrasão, desgaste e corrosão em
peças e componentes metálicos e cerâmicas. Os pós mais usados são Carboneto de
tungsténio, Carboneto de crómio, Aço inoxidável, ligas de níquel, Alumínio,
etc..

Thermal spraying can provide thick coatings (approx. thickness range is 20
micrometers to several mm, depending on the process and feedstock), over a
large area at high deposition rate as compared to other coating processes such
as electroplating, physical and chemical vapor deposition. Coating materials
available for thermal spraying include metals, alloys, ceramics, plastics and
composites. They are fed in powder or wire form, heated to a molten or
semimolten state and accelerated towards substrates in the form of
micrometer-size particles. Combustion or electrical arc discharge is usually
used as the source of energy for thermal spraying. Resulting coatings are made
by the accumulation of numerous sprayed particles. The surface may not heat up
significantly, allowing the coating of flammable substances.

Coating quality is usually assessed by measuring its porosity, oxide content,
macro and micro-hardness, bond strength and surface roughness. Generally, the
coating quality increases with increasing particle velocities.

The utilisation of the HVOF coating technique allows the application of coating
materials such as metals, alloys and ceramics to produce a coating of
exceptional hardness, outstanding adhesion to the substrate material, and
providing substantial wear resistance and corrosion protection.
As the technology specialists in HVOF coating, Bodycote provides an array of
spray coating materials to suit your specific needs. Backed by a
customer-driven service, our facilities process a wide variety of component
sizes to exacting standards with reliable, repeatable results.
