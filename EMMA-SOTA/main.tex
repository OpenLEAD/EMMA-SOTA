%%This is a very basic article template.
%%There is just one section and two subsections.
\documentclass{main}
 

\usepackage{graphicx}      % include this line if your document contains figures
\usepackage{natbib}        % required for bibliography
\usepackage{subfigure}
\usepackage{enumerate}
\usepackage[utf8]{inputenc}
\usepackage{graphicx}
\usepackage{caption}
\usepackage{subcaption}
\usepackage[portuguese]{babel}
\usepackage[none]{hyphenat}
\sloppy

\begin{document}

\begin{frontmatter}

\title{Robôs para Metalização por Chama de Alta Velocidade (HVOF) de palhetas de
turbinas hidráulicas: estudo do estado da arte
\thanksref{footnoteinfo}} 

\thanks[footnoteinfo]{This work is supported by ESBR under contract COPPETEC
JIRAU 151/13 6631-0002/2013 (ANEL R\&D program).}

\author[1]{Renan S. Freitas}
\author[1]{Gabriel Alcantara C. S.}
\author[1]{Eduardo Elael M. S.}
\author[1]{Ramon R. Costa}
\author[2]{Sylvain Joyeux}
\author[2]{Patrick M. Paranhos}

  \address[1]{Departamento de Engenharia Elétrica, COPPE UFRJ, Rio de Janeiro,
  Brasil} 
 %TODO
  \address[2]{Centro de Inovação em Robótica (CIR), Rio de Janeiro, Brasil}
 
\begin{abstract}                % Abstract of not more than 250 words.
%TODO 
\end{abstract} 

\begin{keyword}
%TODO
\end{keyword}

\end{frontmatter}

\section{Introdução}
%TODO Renan: Introdução
Hydropower is the most mature, reliable and cost-effective
renewable power generation technology available \citep{brown}, accouting 16
percent of global electricity generation. The global hydropower use and
capacity will increase about 3.1\% each year for the next 25 years \citep{wi}.
The total investment for large-scale hydropower projects
typically range from USD 1000/kW to around USD 3500/kW and, once commissioned,
the annual operation and maintenance costs of hydropower plants are often
quoted as 4\% of the investment per kW per year \citep{ecofys}. 

In the specific case of Brazil, the third biggest hydroelectric potential of
the world, hydropower represents 84\% of its electric power total production.
Brazil is the second biggest country of installed hydropower capacity, 84 GW,
and in the Amazon basin, in Madeira river, this number will be increased next
years by the construction of Santo Antonio (3150 MW) and Jirau (3300 MW) power
plants. The dependance on this renewable power source mobilizes private
initiative investments on research centers and universities, and motivates the
development systems with a high degree of automation based on advanced robotic
systems \citep{aneel}.

A major challenge for hydropower companies is\ldots


 


In this paper, we present the state of the art in\ldots

%a general overview of the
%ROSA robot, and a detailed description of the embedded electronics, power
% supply system and software architecture. The robot is designed to perform monitoring and inspection
%tasks of the stoplogs' stacking and retrieving process in a power
%plant. Carrying different sensors, the robot analyses sensor data \emph{in
%loco} or stores it for a posterior analysis, interprets the results, and
%sends specific data to the operator. The sensors can identify the lifting beam
%actual operation (stack/retrieve), stoplog attachment/detachment, the
%lifting beam inclination, the system depth in water, and a
%profiling sonar for sediments inspection. 

%This text is organized as follows: the state of the art general overview of the
%robot and its main challenges are presented in Section \ref{sec:sota}, detailed
%descriptions of the embedded electronics, the vehicle support system, power
%supply system, and software architecture are taken in
%Sections \ref{sec:electronics_overview}, \ref{sec:powersupply_overview}, and
%\ref{sec:software} respectively.
%In Section \ref{sec:results}, preliminary results are shown, and concluding
%remarks are drawn in Section \ref{sec:conclusions}.
\section{Considerações Gerais}\label{sec:consideracoes}
\section{Estado da arte}\label{sec:sota}

\input{sota_intro}
\input{RoboTurb}
\subsection{Robôs escaladores}
%TODO características gerais do robo: fixação,
% sensores, sistema HVOF e etc
% aplicação,
% vantagens e desvantagens
Robôs escaladores são sistemas capazes de sustentar seu próprio peso contra a
gravidade, movendo-se em simples ou complexas estruturas geométricas, como
paredes, tetos e telhados, palhetas de turbinas e plantas nucleares.
Essa classe de robôs oferece eficiência operacional em ambientes
de alta periculosidade, sendo utilizados visando saúde e segurança dos
trabalhadores, como em inspeção e limpeza de arranha-céus, diagnóstico de
tanques de armazenamento em plantas nucleares, solda e manutenção de cascos de
navios e palhetas de turbinas \cite{clawar}.

Os grandes desafios nos projetos de sistemas escaladores são mobilidade e
aderência, além de consumo de energia, capacidade de carga e peso. Em
\cite{modular}, os robôs escaladores são divididos em seis tipos de locomoção:
por pernas, como andador, deslizante, com rodas, por esteiras, e avanço
pendurado por braços; e seis categorias de adesão: sucção ou pneumática,
magnética, eletrostática, química, preensão, e híbrido. 

No caso específico deste estudo da arte, podem-se destacar os robôs escaladores
com as seguintes aplicações (\cite{climbsurv}): 

\begin{itemize}
  \item \emph{Construção de navios e turbinas}: RRX3 para soldagem
  (\cite{rrx3}), \emph{Climbing Robot for Grit Blasting} para limpeza
  (\cite{crgb}) e ICM Robot para inspeção \citep{icm};
  \item \emph{Construção industrial}: ROMA II para inspeção \citep{roma} e
  CROMSCI para inspeção \citep{CROMSCI};
 \item \emph{Planta nuclear}: Robug IIs para manutenção \citep{robug}; 
 \item \emph{Planta petroquímica}: ROBICEN para inspeção \citep{robicen} e
  TRIPILLAR para inspeção \citep{tripillar}.  
\end{itemize}

 Hydro Electric Dam in Virgina.

que operam com instrumentos para soldagem e com grande capacidade de mobilidade: ROMA II, Robug II, Roboclimber, Robug IIs, RRX, REST 2,

\input{RIWEA}
\subsection{Manipulador com base esférica}
%TODO características gerais do robo: fixação,
% sensores, sistema HVOF e etc
% aplicação,
% vantagens e desvantagens
\input{sota_concl}

\section{Projeto de robô autônomo para HVOF}\label{sec:projeto}

% TODO soluções e dificuldades comuns
O projeto de robôs autônomos para HVOF em pás de turbinas hidráulicas contempla
as soluções que atendem a \textbf{todos} os requisitos da aplicação. Dessa
forma, serão idealizados robôs com a fusão das tecnologias expostas na
seção~\ref{sota}.
 
\subsection{Acesso pela escotilha de dimensão pequena}
%TODO Elael: prós e contras do acesso, soluções: manipuladores
% industriais, manipuladores customizados, trilhos. Incluir figuras do
% posicionamento das pás e cálculos do tamanho mínimo do manipulador

\subsection{Acesso pela escotilha de dimensão grande}
%TODO Renan: prós e contras do acesso, soluções: revisar e adaptar os sistemas
% já descritos, apresentar soluções de logística para a solução


 % attach a rail to the blade and move it manually
 
 % attach a rail one the nose and ground, 1D movement and move the blade to
 

\subsection{Projeto de robôs escaladores}\label{proj_climbers}

Nesta subseção, consideram-se soluções para HVOF de pás de turbinas robôs
escaladores com fusão das tecnologias documentadas na
seção~\ref{sota}, subseção~\ref{sota_climbers}. Duas soluções serão abordadas:
a primeira será uma versão adaptada do robô \emph{The Climber}, ICM, e a
seguinte a fusão das tecnologia do Roboturb e o \emph{Climbing robot for Grit
Blasting}.

\textbf{Versão adaptada The Climber}  

O robô \emph{The Climber}, ICM, é uma solução comercial que atende muitas das
especificações HVOF e possibilita aperfeiçoamento sem comprometer sua
estrutura. O robô possui sistema de adesão por sucção e locomoção através de
esteiras flexíveis. O sistema já foi testado em ambientes de alta
periculosidade, como turbinas eólicas, usinas hidrelétricas e outros, porém
deve-se ter conhecimento preciso das \textbf{características do ambiente,
compartimento da turbina bulbo, como temperatura e umidade}. Podemos dividir o
projeto em quatro sistemas: locomoção, adesão, manipulador e autonomia.

O sistema desenvolvido em \cite{kim2008development} têm mecanismos de
locomoção por esteiras e adesão por sucção compostos por polias, correias de
borracha, ventosas, válvulas para cada ventosa, motores DC para as polias,
sistemas de controle para as válvulas e para os motores . \emph{The Climber}
utiliza apenas uma câmara de vácuo, em vez de ventosas, e esteiras flexíveis que
permitem maior suavidade e continuidade ao movimento. A solução por uma única
câmara parece mais vantajosa, já que o robô consegue se locomover em curvaturas
de até 30 cm de raio.

No caso específico da aplicação HVOF, o processo é realizado com
manipulador enquanto o robô percorre a pá da turbina. A locomoção do
robô sob a pá levanta algumas questões de projeto: \textbf{a
temperatura da turbina durante o procedimento pode inviabilizar a solução devido ao contato
contínuo robô/turbina}; deve ser considerado o \textbf{tempo de cura do HVOF
para a locomoção sob a pá}; verificar se há \textbf{acesso a todos os
pontos da pá, limites em relação à parede, disposição da pá, sobreposição entre
pás}; e \textbf{como se comporta o robô em curvaturas acentuadas, e
o quanto a pá pode ser girada para facilitar locomoção}. A viabilidade de
qualquer projeto com robôs escaladores dependem dessas questões, independente da
tecnologia de locomoção e adesão.

O sistema de adesão ativo por sucção deve considerar a \textbf{força máxima que
pode ser realizada sem danificar o revestimento ou a superfície da pá} e
\textbf{o peso total do conjunto do sistema HVOF}. Além disso, um mecanismo
inteligente de segurança, possivelmente utilizando acelerômetros e outros
sensores, deve ser embarcado no robô para garantir o \textbf{desligamento do
sistema (tempo a ser estimado)} em caso de queda ou falha, dependendo das
\textbf{condições de segurança}. A solução de um robô com planejamento de
trajetória aumenta a segurança da operação e o controle ótimo do mecanismo de
adesão pode limitar a força máxima de sucção.

O manipulador a ser projetado para aplicação HVOF deve possuir as seguintes
características: ser leve para não comprometer a adesão e equilíbrio do sistema
móvel; \textbf{rápido e preciso conforme requer a aplicação HVOF}; modular, já
que a operação será realizada in-situ e a sua montagem deverá ser realizada
caso o \textbf{acesso ao local seja incompatível com as dimensões do robô com
manipulador}; não é necessário possuir grandes dimensões, pois o robô é móvel e
pode percorrer a pá, porém deve ser suficiente para operar em \textbf{pontos de
difícil acesso à base} e considerar a \textbf{distância mínima entre pistola
HVOF e pá}; e ser capaz de sustentar a força gerada pela pistola HVOF. A
literatura sobre manipuladores é bastante consolidada, sendo muitos dos
problemas citados já resolvidos e disponíveis no mercado, como o desenvolvido
em \cite{manzdevelopment}.

O manipulador também executará a função de remoção de material, preparação para
HVOF e medida de integridade da pá. Há diversos meios para remoção do
revestimento, como laser, abrasão e outros. O manipulador poderá ser modificado
de acordo com o procedimento (remoção ou aplicação do revestimento) ou um outro
sistema pode ser utilizado, dependendo de \textbf{como é realizado o
procedimento de remoção}. A tecnologia que verifica a necessidade de
revestimento, com sensores laser e ultrassom, e poderá indicar o \textbf{mapa
ou apenas realizar um teste de sucesso/falha} \citep{escaler2006detection}.

O sistema autônomo de um robô móvel é a inteligência do robô. Ele abrange o
controle de missão, ou seja, o planejamento e execução das tarefas.
A locomoção será realizada pelo controle dos motores em conjunto com o controle do
sistema ativo de adesão por sucção, o planejamento de trajetória, desvio de
obstáculos e mapeamento do ambiente, através de um conjunto de sensores, como
laser e acelerômetros. O controle do manipulador poderá ser cinemático por
servovisão ou pela estruturação do ambiente. E um sistema de suporte do veículo
icará responsável pela segurança, bom funcionamento e gerenciamento de potência
do robô.

As características descritas acima como solução de um robô escalador impede a
troca automática entre pás. Um robô escalador com tecnologia de avanço pendurado
por braços é uma solução muito custosa em termos de controle e estrutura
mecânica. Outra solução seria um robô com locomoção por segmentos deslizantes,
como o RRX3, e adesão por sucção, porém a flexibilidade exigida para a locomoção
entre pás e a distância entre turbinas complexifica o projeto. Dessa forma, a
troca entre pás deverá ser manual e, caso o \textbf{cone da turbina também seja
revestido por HVOF}, o operador deve manualmente transportar o robô.

\textbf{Versão adaptada Roboturb}

O Roboturb, como já descrito na subseção~\ref{sec::rail}, é um
manipulador que se locomove em um trilho, este acoplado à pá da turbina
por ventosas (sucção). A solução não permite a extensão
do manipulador, já que o peso desequilibra a estrutura e não há torque para
compensar a força exercida no efetuador durante a operação HVOF. A segunda
solução de robôs escaladores é adicionar um trilho perpendicular e transformar
o Roboturb em um robô móvel, com locomoção através de dois trilhos, idéia
semelhante ao \emph{Climbing robot for Grit Blasting}, que utiliza duas
plataformas deslizantes com ventosas.

Os trilhos são compostos por esteiras flexíveis nas extremidades para a
locomoção, como \emph{The Climber}, e as ventosas são ativas e distribuídas por
todo o trilho. O manipulador só necessitaria mover em um dos trilhos para
percorrer toda a pá, já que os trilhos também se movimentam. 

A solução de trilhos móveis com manipulador é dependente à curvatura da pá da
turbina e o aumento da flexibilidade do trilho para se locomover sob a pá pode
impedir a movimentação do manipulador. Dessa forma, é considerada uma solução
muito específica e restrita à aplicação. 
% RIWEA-like rail on which the coating system
%  place blade horizontal and use wheels, against the blade's rim,to move a rail
  % along the blade's shape
  %  two-arm solution (or one arm + magnetic attachment point) to allow reaching
  % the top of the blade from the very bottom

\subsection{Acesso pela jusante}
%TODO Abelha: prós e contras do acesso, soluções, apresentar soluções de
% logística


Como última opção de acesso ao rotor existe a possibilidade de utilização do
tubo de sucção ou descarga como meio de entrada à turbina. Com o fluxo de água
parado, é possível utilizar o Rio como meio de lançamento do sistema. A
complexidade da operação para utilizar esse acesso é maior, entretanto existem
vantagens que podem tornar essa solução possível e mais atrativa.

\textbf{Vantagens}
\begin{itemize}
  \item Virtualmente nenhuma restrição de tamanho
  \item Flexibilidade de soluções
  \item Facilidade de utilização de um manipulador industrial \textit{standard}
  \item Possibilidade de implementação em outras usinas
\end{itemize}

\textbf{Desvantagens}
\begin{itemize}
  \item Complexidade de lançamento e recuperação
  \item Custo
  \item Possibilidade de correnteza
  \item Complexidade logística de transporte entre a entrada do tubo de sucção e
  o aro câmara
  \item Complexidade de prototipação
\end{itemize}

As possíveis soluções foram divididas nas etapas necessárias para a operação, ou
seja, lançamento e recuperação do sistema, logística de transporte e o robô de
metalização propriamente dito.

Para esse acesso, o maior obstáculo presente é o desenvolvimento de um sistema
de lançamento e recuperação do robô, a partir do rio, até o interior da turbina.
Essa operação deverá ser realizada com a turbina alagada e, em seguida, deverá
ser realizada a drenagem da mesma. É importante que o sistema de lançamento seja
robusto e garanta o perfeito posicionamento do robô dentro da turbina, assim
como, a sua recuperação. Uma vez que o sistema não pode se perder no leito do
rio.

Primeiramente, o sistema deve ser a prova d'água com classificação de pelo
menos 50m.
Sendo assim, um vaso de pressão para o transporte do robô até o interior da turbina deve ser
desenvolvido, não havendo necessidade do maniupuladore responsável pela
metalização em si ser a prova d'água. O \textit{container} de transporte
submarino deve ser menor que o tamanho do vão do stoplog, uma vez que ele
utilizirá esse caminho para acessar o tubo de descarga. Por outro lado, deve ser
grande o suficiente para o robô e todo o material necessário seja transportado e
também que suporte uma escotilha de acesso de tamanho suficiente para que todo o
sistema seja retirado do seu interior.

Para o sistema de lançamento foi deslumbrada uma estrutura de transporte que
utilizará o pórtico rolante e o trilho guia dos stoplogs. Após a submersão da
estrutura, um mecanismo de lançamento, inspirado em um paletizador, é
responsável pelo posicionamento do \textit{container} sempre no mesmo ponto em
relação ao tubo de descarga. Com o vaso de pressão posicionado, a turbina deve
ser, então, drenada. Com a turbina seca, o robô pode ser retirado de seu
envólucro e a operação de metalização pode ter seu início. Uma etapa crítica da
operação é a recuperação do sistema, na qual a turbina deve ser novamente
alagada e os os stoplogs retirados. Em seguida, a estrutura de transporte deve
recuperar o \textit{container} transportador na mesma posição em que o sistema
foi lançado. O sucesso dessa operação tem como \textbf{hipótese que a velocidade
de drenagem e a correnteza gerada por essa operação não são suficientes para
retirar o container (mais pesado que a água) de sua posição inicial}. 

A movimentação do robô do ponto de lançamento até o aro câmara deverá ser
realizada a partir da utilização de cordas, roldanas e talhas. Caso necessário,
pode ser desenvolvido um sistema de locomoção com trilhos e/ou rodas atuadas
para o posicionamento automático do robô.

O robô de metalização pode ter diversos fomatos, mas devido a possilidade de se
utilizar um manipulador industrial padrão, o projeto inicial consiste em uma
base de apoio e um manipulador com alcance para o processmento de uma face da pá
posicionado de frente para pá, ou um manipulador posicionado entre duas pás com
alcance de para processar as duas as faces das pás voltadas para ele.

\subsubsection{Dimensionamento da base}

Para manipuladores com longo alcance, as forças e torques envolvidos requerem
uma estrutura de fixação do robô de forma que o sistema como um todo não se
movimente e, no caso extremo, tombe. Normalmente, os manipuladores robóticos são
fixados no chão e as características da superfície e dos parafusos são
estipulados pelo fornecedor a partir dos valores máximos de torque e força que o
manipulador exerce em sua base. Para uma base apoiada no chão, dois fatores
influenciam capacidade de estabilização da estrutura: o raio da base e o seu
peso.

O raio da base $r_{b_f}$ é limitado pelo ambiente da turbina e para cada
posicionamento existem restrições específicas. 
Para a realização dos cálculos de dimensionamento foi considerado,
primeiramente, o manipulador posicionado em frente a pá e processando somente uma face por vez
e a uma altura de 1000mm, posição em que o alcance necessário do robô é de
1800mm. Para essas características, o tamanho máximo que a base pode assumir é
de aproximadamente 1600mm, caso a estrutura seja projetada de forma a seguir os contornos do aro câmara.
A figura \ref{fig::base_aro_frente} %TODO figura base_aro_frente
representa um esboço da vista frontal do aro câmara e a largura máxima que a
base pode assumir. A análise da dimensão máxima da base no sentido paralelo ao
fluxo da água pode ser realizada com o auxílio do desenho técnico fornecido pela
ESBR, ilustrado na figura \ref{fig::turbine_side}. O limite superior nessa
região é determinado pela transição do aro câmara para a região inclinada do
tubo de sucção e é de aproximadamente 1600mm. Entrentanto esse limite pode ser
contornado construindo-se um plano horizontal ou projetando-se a base de forma que ela acompanhe essa
inclinação. Ao se incluir o dimensionamento da base no cálculo do alcance mínimo
do manipulador deve ser realizado uma alteraçao, uma vez que agora o manipulador
se encontra deslocado da superfície da pá. Sendo assim, alcance mínimo se
relaciona com o tamanho do raio da base de acordo com
$$a_{min}=\sqrt{r_b^2+1800^2}.$$

O cálculo das dimensões da base com o robô posicionado dentro do aro câmara e
entre as pás depende do angulo de ataque das pás e do cálculo do ângulo diédrico
entre elas. A amplitude do movimento de rotação $alpha$ das pás é de $14,5^o$
para cada lado a partir da posição zero, entretanto \textbf{essa posição não pôde ser
informada no momento da viagem de reconhecimento e ainda não foi
disponibilizada}. Para critério de cálculos foi utilizado um ângulo de
$45^o$ como a posição de maior abertura das pás e o zero foi considerado como
a reta perpendicular ao fluxo de água. O ângulo diédrico $\theta$ entre as pás depende da rotação
sofrida pelas mesmas e obedece a relação $\cos{\theta} = \sin^2{\alpha}.$

O arco de circunferência pode ser obtido a partir da relação $arc=R*\alpha$ com
R=3850mm. O raio máximo da base pode ser calculada como 
$$r_{b_e} = (R - h_{b_e})\tan{\theta/2},$$  ilustrado na figura
\ref{fig:calc_base_entre} e com $h_{b_e}$ sendo a altura da base.

O peso mínimo que a base do robô deve possuir está diretamente relacionada com o
tamanho de seu raio. A firgura \ref{fig::tilt_robot} faz uma representação
simplificada da forma que o torque de capotamento máximo atua no robô e em sua
base. Na situação limite, considerando o torque com sentido horário, a força
normal entre a base e a superfície de apoio $N_2$ teria módulo igual a zero.
Considerando o pior caso, ou seja, a força vertical que o robô exerce na base é
composta apenas pelo seu peso $W$ para que a base não se mova durante a
operação, temos que o somatório das forças e torques sejam iguais a zero.

A análise das forças nos fornece que $N_1$ tenha módulo igual ao peso do robô e
o somatório dos torques se reduz a $M_k-Wr_b=0$. Sendo assim, a relação entre
o raio da base, seu peso e o torque máximo de capotamento exercido pelo robô é
da forma 

$$M_k=Wr_b.$$

%TODO refazer figura
\begin{figure}[h!]
\centering
	\includegraphics[width=0.5\columnwidth]{figs/base/tilt}
	\caption{Forças e torques máximos entre o robô e sua base.}
\end{figure}

Uma vez que a superfície do aro câmara e a região adjacente no tubo de sucção
são \textbf{ferromagnéticas}, é possível a utilização de bases magnéticas para
uma compensação do peso e raio necessário para a estabilização do robô. Os
dispositivos magnéticos se dispõem de duas maneiras para essa aplicação:
eletretromagnéticos e imãs permanentes. O primeiro caso tem como principal
vantagem a possibilidade de acionamento remoto, entretanto para situações de
falha em que haja perda de fornecimento de energia a força de atração também é
perdida. O segundo caso consiste em imãs permanentes arrumados de maneira que
seja possível organizar o fluxo magnético e, assim, controlar por meio de uma
alavanca a presença ou ausência de força magnética. A figura
\ref{figs/base/imas} ilustra os dois tipos de bases magnéticas citados.
Comercialmente, foram encontrados bases magnéticas com capacidade de até 3000N.

%TODO figura imãs













\section{Compilação de questões}\label{sec:questoes}
Nesta seção, será apresentada a compilação das questões em aberto até o momento,
sendo de extrema necessidade a elucidação para a escolha da solução mais
apropriada e eficiente à aplicação.

\textbf{Questões em aberto:}
\begin{itemize}
	\item Tamanho e peso do mecanismo de metalização. 
	\item Espaço de trabalho disponível.
	\item Distância que o efetuador deve manter da superfície da pá.
	\item Resistência do material da superfície da pá à sucção e magnetismo.
	\item Tempo resfriamento ou fixação do material após HVOF.
	\item Características do ambiente: temperatura e umidade.
	\item Temperatura da pá da turbina durante o procedimento HVOF.
	\item Acesso a todos os pontos da pá, limites em relação à parede, disposição
	da pá e sobreposição entre pás.
	\item Quanto a pá pode ser girada para facilitar a locomoção de robôs
	escaladores.
	\item Tempo necessário para desligamento completo do sistema HVOF.
	\item Condições de segurança do ambiente.
	\item Rapidez e precisão do manipulador para sistema HVOF.
	\item Como é realizado o procedimento de remoção do revestimento.
	\item Durante o procedimento de verificação de cavitação, deverá ser gerado um
	mapa ou apenas realizar um teste de sucesso/falha.
	\item O cone da turbina também é revestido.
\end{itemize}



\section{Conclusão e trabalhos futuros}\label{sec:conclusions}

O estudo de viabilidade técnica se mostrou promissor até o momento, de
forma\ldots

  
\bibliography{main} 
\appendix
\end{document}
