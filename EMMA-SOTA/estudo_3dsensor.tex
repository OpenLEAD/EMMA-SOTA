\subsection{Estudo de sensores 3D disponíveis}

O processo de metalização utilizado atualmente considera que a posição e
orientação da pá é fixa em relação ao robô e, uma vez, que corretamente
posiciona, o processo é executado em malha aberta. Entretanto, para qualquer uma
das soluções propostas por esse documento, não é possível assumir que nem a
posição nem a orientação do manipulador, em relação a pá a ser processada, se
manterão fixas.

Para um correto planejamento de trajetória que o manipulador deve seguir durante
a tarefa de metalização, é importante o conhecimento da transformada entre o
sistema de coordenada do manipulador e da pá a ser processada. Portanto, é
necessário utilizar algum sistema que possibilite a aquisição de informações a
respeito do ambiente e da posição relativa entre o manipulador e as pás.

A utilização de um sensor de aquisição de dados espaciais não se limita somente
a localização, mas, dependendo do sistema a ser escolhido, pode também ser útil
na reconstrução do modelo do perfil hidráulico da pá, tanto do perfil ideal
quanto do estado atual da pá a ser processada (Tarefas descritas em
\ref{desc_taref}).

Esta seção irá apresentar os segmentos de sensores capazes de suprir essa
necessidade, assim como suas vantagens e limitações. 


\subsubsection{3D scanners}

3D scanners são equipamentos de alta precisação utilizados na indústria
geralmente em aplicações de metrologia, construção civil, monitoramento de
deformações, entre outras. O equipamento consiste em um feixe de laser que é
direcionado por meio de um espelho e a partir da mudança de fase do sinal
refletido é possível calcular a distância até o objeto atingido.

%TODO LEMBERAR QUE EXISTE LASER TRIANGULATION TAMBÉM!


%TODO exemplos dos sensores de 3D scanners
%TODO Pros e cons

\subsubsection{ToF Cameras}

As conhecidas como Time-of-Flight são dispositivos compostos por apenas uma
câmera, não necessitando de uma configuração stereo para triangularização de
imagens. Esse tipo de dispositivo utiliza uma fonte infra-vermelho interna e de
forma análoga aos dispositivos laser, calcula a distância a partir da diferença
de fase do sinal refletido. Entretanto, essa tecnologia possibilita o cálculo
simultâneo das distâncias de cada objeto na região iluminada pela fonte IR,
mesmo que com resoluções limitadas.

%TODO exemplos dos sensores de ToF

Essa tecnologia tem como vantagem o tamanho relativamente compacto dos sensores,
não precisa de calibração extrínseca e também não é muito sensivel a iluminação
presente no ambiente, pois possui fonte de iluminação própria.

%TODO Pros e cons

\subsubsection{Câmeras de Luz Estruturada}

Estes sensores constituem de uma fonte emissora de infra-vermelho e um receptor.
Um padrão é projetado na cena a ser reconstruida e a partir da distorção desse
padrão é possível o cálculo de distâncias. 

%TODO exemplos dos sensores de luz estruturada
%TODO Pros e cons

%TODO ELAEL - decidir se abre uma subseção d eaplicações ou coloca um exemplo de
% aplicação em cada componente - utilizar o seu material do SOTA em 3D sensors. 


