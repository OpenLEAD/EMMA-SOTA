\subsection{Robôs sobre trilhos}
%TODO características gerais do robo: fixação,
% sensores, sistema HVOF e etc
% aplicação,
% vantagens e desvantagens

%oq sao e motivaçaõ
%teconologias de fixaçaõ
%robos
%vantagems/desvantagens

Na indústria, a automatização de processos de \textit{hard coating},é
normalmente realizada com a utilização de manipuladores robóticos. Esse tipo de
robô proporciona uma versatilidade operações, uma vez que podem ser
reprogramados para realizar tarefas repetidamente com alta precisão. O espaço de
trabalho de um braço robótico é dependente do numero de juntas presentes no
robô, rotacionais ou prismáticas, e pelo tamanho de cada elo. As juntas
determinam os graus de liberdade e mobilidade que um manipulador possuí,
enquanto o tamanho dos elos influenciária no alcane máximo do robô.

Os processos de manunteção de turbinas, como o \textit{hard coating}, exigem a
realização trajetórias complexas, o que siginifica a necessidade de várias
juntas, e o tamanho das pás da turbina exigem que o manipulador tenha um alcance
máximo pelo menos maior que a maior dimensão da pá. Portanto, os manipuladores
robóticos utilizados para esse tipo de aplicação são, geralmente, robôs
industrias que pesam centenas de quilos, com mais de 2 metros de alcance máximo
e necessitam de uma fixação que garantam que o manipulador não irá se movimentar
ao realizar suas tarefas programadas. 

Para a automatização de um processo de manutenção \textit{in-situ} é
necessário, porém, que o sistema projetado seja compacto, de fácil instalação, e
que nâo exija alterações estruturais permanentes para a sua utilização. Um
manipulador robótico convencional que necessite de um espaço de trabalho
suficiente para executar tarafas em toda a superfície da pá de uma turbina se
torna, então, inviável. 

Uma estratégia para reduzir o tamanho e peso de um manipulador robótico é
torná-lo móvel. Isso pode ser alcançado com a introdução de mais uma junta
prismática acoplada a um trilho, possibilitando que o manipulador estenda seu
espaço de trabalho por toda a extensão do trilho.

Na literatura foram encontradas duas soluções para aplicações de manutenção e
inspeção, como solda. As aplicações diferem principalmente na estratégia de
fixação do trilho. O Roboturb \cite{roboturb} realiza a fixação diretamente na
pá do rotor, enquanto o robô Scompi \cite{scompi} utiliza uma fixação exterior 


