\subsection{Robôs sobre trilhos}
%TODO características gerais do robo: fixação,
% sensores, sistema HVOF e etc
% aplicação,
% vantagens e desvantagens

%oq sao e motivaçaõ
%teconologias de fixaçaõ
%robos
%vantagems/desvantagens

Na indústria, a automatização de processos de metalização,é
normalmente realizada com a utilização de manipuladores robóticos, pois oferem a
versatilidade de tarefas e espaço de trabalho possíveis necessários para esse
tipo de aplicação. Entretanto, um sistema composto por um braço robótico capaz
de operar em toda a extensão da superfície de uma pá de uma turbina hidrelétrica
não é compacto nem móvel o suficiente para ser instalado e desinstalado para a
operação de manuntenção \textit{in-situ}.

Uma estratégia para reduzir o tamanho e peso de um manipulador robótico é
torná-lo móvel. Isso pode ser alcançado com a introdução de mais uma junta
prismática acoplada a um trilho, possibilitando que o manipulador estenda seu
espaço de trabalho por toda a extensão do trilho.

Na literatura foram encontradas duas soluções para aplicações de manutenção e
inspeção, como solda. As aplicações diferem, principalmente, na estratégia de
fixação do trilho. O Roboturb \cite{roboturb} realiza a fixação diretamente na
pá do rotor, enquanto o robô Scompi \cite{scompi} utiliza um trilho fixado em
estruturas adjacentes à pá ou peça a ser reparada.

O Roborturb consiste em um manipulador robótico com seis juntas de revolução e
uma junta primsática acoplada a um trilho flexível, como pode ser observado
na figura \ref{fig::roboturb}, utilizado para o preenchimento de cavidades
geradas por cavitação.
O trilho pode ser conformado e, então, fixado à superfície da pá por meio de um
 sistema passivo de ventosas ou imãs.O robô tem 
a possibilidade de utilizar dois efetuadores distintos, o primeiro consiste em 
um sensor ótico para inspeção do estado de erosão da pá e o segundo consiste em
uma ferramenta de solda do tipo tocha plasma PWH-4A com alimentador automático
de arame, responsável pelo depósito de solda para o preenchimento das cavidades
identificadas pelo sistema.

\begin{figure}[!htb]
	\centering
    \begin{minipage}[b]{0.45\linewidth}	
		\includegraphics[width=\columnwidth]{figs/trilhos/roboturbpaper}
		\caption{Roboturb - Manipulador robótico sobre trilho flexível}
		\label{fig::roboturb}
	\end{minipage}
	\begin{minipage}[b]{0.45\linewidth}
		\includegraphics[width=\columnwidth]{figs/trilhos/scompi}
		\caption{SCOMPI - Manipulador robótico sobre trilhos rígidos}
		\label{fig::scompi}
	\end{minipage}
\end{figure}
%TODO ajeitar figura

Por sua vez, o robô Scompi, fig \ref{fig::scompi}, é um manipulador
multipropósito projetado para realizar reparos em turbinas do tipo \textit{Francis}, como solda e
esmerilhamento das pás. O sistema possuí seis graus de liberdade, sendo 
consitituído por um braço robótico com cinco juntas derevolução e o último grau
 de liberdade proveniente de uma junta prismática quepercorre um sistema de 
 trilhos retos ou curvos. 


Sistemas baseados em trilhos tem como maior benefício a redução do tamanho e,
consequentemente, do peso do manipulador necessário para a execução de tarefas
em um espaço de trabalho que englobe toda a superfície da pá a ser reparada.
Essa redução proporciona facilidade de transporte do robô até o interior da
turbina e também o projeto de manipuladores que tenham a rigidez necessária para
a realização das tarefas desejadas, uma vez que manipuladores simples com
rigidez e espaço de trabalho necessários seriam muito pesados. Entretanto,
sistemas baseados em trilhos com fixação na própria pá do rotor, necessitam que
o trilho seja movidos caso toda a superfície da pá necessite sofre manuntenção,
uma vez que o a área em que o trilho está apoiado não pertence ao espaço de
trabalho do robô. Em adição, sistemas com fixação de trilhos nas estruturas
adjacentes à pá devem atentar as condições para a instalação dispostas pelo
ambiente para equilibrar a relação de custo benefício entre facilidade de
instalação e remoção do trilho e a robustez do mesmo, uma vez que sistemas
permanentes dificilmente são possíveis no contexto das aplicações citadas.



