\section{Compilação de questões}\label{sec:questoes}
Nesta seção, será apresentada a compilação das questões em aberto até o momento,
sendo de extrema necessidade a elucidação para a escolha da solução mais
apropriada e eficiente à aplicação.

\textbf{Questões em aberto:}
\begin{itemize}
	\item Tamanho e peso do mecanismo de metalização. 
	\item Espaço de trabalho disponível.
	\item Distância que o efetuador deve manter da superfície da pá.
	\item Resistência do material da superfície da pá às forças de sucção e
	magnetismo.
	\item Tempo de resfriamento ou fixação do material de revestimento após HVOF.
	\item Características do ambiente: temperatura e umidade.
	\item Temperatura da pá da turbina durante o procedimento HVOF.
	\item Acesso a todos os pontos da pá, limites em relação à parede, disposição
	da pá e sobreposição entre pás.
	\item Quanto a pá pode ser girada para facilitar a locomoção de robôs
	escaladores.
	\item Geometria da pá.
	\item Tempo necessário para desligamento completo do sistema HVOF.
	\item Condições de segurança do ambiente.
	\item Rapidez e precisão do manipulador para sistema HVOF.
	\item Como é realizado o procedimento de remoção do revestimento.
	\item Durante o procedimento de verificação de cavitação, deverá ser gerado um
	mapa ou apenas realizar um teste de sucesso/falha.
	\item O cone da turbina também é revestido.
	\item Há necessidade a automatização do processo de preparação e
inspeção da pá.
	\item Especificações de cada processo a ser realizado pelo robô: forças
	relacionadas a cada processo, complexidade das trajetórias e as restrições de movimentos impostas pelo método aplicado.
	\item A turbina pode ser girada.
	\item A distância máxima da superfície da pá até a base do cone da turbina.
\end{itemize}


