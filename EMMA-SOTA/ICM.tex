\section{Introdução}
Hydropower is the most mature, reliable and cost-effective
renewable power generation technology available \citep{brown}, accouting 16
percent of global electricity generation. The global hydropower use and
capacity will increase about 3.1\% each year for the next 25 years \citep{wi}.
The total investment for large-scale hydropower projects
typically range from USD 1000/kW to around USD 3500/kW and, once commissioned,
the annual operation and maintenance costs of hydropower plants are often
quoted as 4\% of the investment per kW per year \citep{ecofys}. 

In the specific case of Brazil, the third biggest hydroelectric potential of
the world, hydropower represents 84\% of its electric power total production.
Brazil is the second biggest country of installed hydropower capacity, 84 GW,
and in the Amazon basin, in Madeira river, this number will be increased next
years by the construction of Santo Antonio (3150 MW) and Jirau (3300 MW) power
plants. The dependance on this renewable power source mobilizes private
initiative investments on research centers and universities, and motivates the
development systems with a high degree of automation based on advanced robotic
systems \citep{aneel}.

A major challenge for hydropower companies is\ldots


 


In this paper, we present the state of the art in\ldots

%a general overview of the
%ROSA robot, and a detailed description of the embedded electronics, power
% supply system and software architecture. The robot is designed to perform monitoring and inspection
%tasks of the stoplogs' stacking and retrieving process in a power
%plant. Carrying different sensors, the robot analyses sensor data \emph{in
%loco} or stores it for a posterior analysis, interprets the results, and
%sends specific data to the operator. The sensors can identify the lifting beam
%actual operation (stack/retrieve), stoplog attachment/detachment, the
%lifting beam inclination, the system depth in water, and a
%profiling sonar for sediments inspection. 

%This text is organized as follows: the state of the art general overview of the
%robot and its main challenges are presented in Section \ref{sec:sota}, detailed
%descriptions of the embedded electronics, the vehicle support system, power
%supply system, and software architecture are taken in
%Sections \ref{sec:electronics_overview}, \ref{sec:powersupply_overview}, and
%\ref{sec:software} respectively.
%In Section \ref{sec:results}, preliminary results are shown, and concluding
%remarks are drawn in Section \ref{sec:conclusions}.