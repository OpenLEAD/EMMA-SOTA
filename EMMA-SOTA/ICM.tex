\subsection{Robôs escaladores}
%TODO características gerais do robo: fixação,
% sensores, sistema HVOF e etc
% aplicação,
% vantagens e desvantagens
Robôs escaladores são sistemas capazes de sustentar seu próprio peso contra a
gravidade, movendo-se em simples ou complexas estruturas geométricas, como
paredes, tetos e telhados, palhetas de turbinas e plantas nucleares.
Essa classe de robôs oferece eficiência operacional em ambientes
de alta periculosidade, sendo utilizados visando saúde e segurança dos
trabalhadores, como em inspeção e limpeza de arranha-céus, diagnóstico de
tanques de armazenamento em plantas nucleares, solda e manutenção de cascos de
navios e palhetas de turbinas \cite{clawar}.

Os grandes desafios nos projetos de sistemas escaladores são mobilidade e
aderência, além de consumo de energia, capacidade de carga e peso. Em
\cite{modular}, os robôs escaladores são divididos em seis tipos de locomoção:
por pernas, como andador, deslizante, com rodas, por esteiras, e avanço
pendurado por braços; e seis categorias de adesão: sucção ou pneumática,
magnética, eletrostática, química, preensão, e híbrido. 

No caso específico deste estudo da arte, podem-se destacar os robôs escaladores
com as seguintes aplicações (\cite{climbsurv}): 

\begin{itemize}
  \item \emph{Construção de navios e turbinas}: RRX3 para soldagem
  (\cite{rrx3}), \emph{Climbing Robot for Grit Blasting} para limpeza
  (\cite{crgb}) e ICM Robot para inspeção \citep{icm};
  \item \emph{Construção industrial}: ROMA II para inspeção \citep{roma} e
  CROMSCI para inspeção \citep{CROMSCI};
 \item \emph{Planta nuclear}: Robug IIs para manutenção \citep{robug}; 
 \item \emph{Planta petroquímica}: ROBICEN para inspeção \citep{robicen} e
  TRIPILLAR para inspeção \citep{tripillar}.  
\end{itemize}

 Hydro Electric Dam in Virgina.

que operam com instrumentos para soldagem e com grande capacidade de mobilidade: ROMA II, Robug II, Roboclimber, Robug IIs, RRX, REST 2,
