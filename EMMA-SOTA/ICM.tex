\subsection{Robôs escaladores}
%TODO características gerais do robo: fixação,
% sensores, sistema HVOF e etc
% aplicação,
% vantagens e desvantagens
Robôs escaladores são sistemas capazes de sustentar seu próprio peso contra a
gravidade, movendo-se em simples ou complexas estruturas geométricas, como
paredes, tetos e telhados, palhetas de turbinas e plantas nucleares.
Essa classe de robôs oferece eficiência operacional em ambientes
de alta periculosidade, sendo utilizados visando saúde e segurança dos
trabalhadores, como em inspeção e limpeza de arranha-céus, diagnóstico de
tanques de armazenamento em plantas nucleares, solda e manutenção de cascos de
navios e palhetas de turbinas \cite{clawar}.

Os grandes desafios nos projetos de sistemas escaladores são mobilidade e
aderência, além de consumo de energia, capacidade de carga e peso. Em
\cite{modular}, os robôs escaladores são divididos em seis tipos de locomoção:
pernas, como andador, utilizando segmentos deslizantes, rodas, esteiras, e
avanço pendurado por braços; e seis categorias de adesão: sucção ou pneumática,
magnética, eletrostática, química, preensão, e híbrido.

No caso específico deste estudo da arte, destacam-se os robôs escaladores com as
seguintes aplicações (\cite{climbsurv}):

\begin{itemize}
  \item \emph{Construção de navios e turbinas}: RRX3 para soldagem
  \citep{rrx3}, \emph{Climbing Robot for Grit Blasting} para limpeza
  \citep{crgb} e ICM Robot para inspeção \citep{icm};
  \item \emph{Construção industrial}: ROMA II \citep{roma} e
  CROMSCI \citep{CROMSCI}, ambos para inspeção;
 \item \emph{Planta nuclear}: Robug IIs para manutenção \citep{robug}; 
 \item \emph{Planta petroquímica}: ROBICEN \citep{robicen} e
  TRIPILLAR \citep{tripillar}, ambos para inspeção.  
\end{itemize}

O RRX3 é um robô com adesão por preensão, locomoção transversal utilizando
segmentos deslizantes e locomoção longitudinal por rodas. Possui um manipulador
com três juntas prismáticas e três juntas de revolução (3P3R) para a operação de
soldagem. O robô tem capacidade de carga acima de 120 kg, possui manipulador
1.5 m de comprimento e 5 kg de carga, e realiza a tarefa de soldagem, que exige
precisão e robustez. Porém, sua locomoção transversal é bem restrita, impedindo
a ação em um ambiente com estrutura geométrica complexa, como a pá da turbina.







 Hydro Electric Dam in Virgina.
