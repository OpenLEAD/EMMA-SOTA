
 % attach a rail to the blade and move it manually
 
 % attach a rail one the nose and ground, 1D movement and move the blade to
A utlização de um manipulador robótico sobre trilhos tem a capacidade de
satisfazer todos os requisitos, até agora observados, para a realização de um
processo de inspeção e metalização utilizando a técnica HVOF. O desenvolvimento
de um sistema compacto para o transporte através dos dutos de acesso e instalção
no camara da turbina é possível, pois o tamanho necessário do manipulador pode
ser reduzido com a mobilidade proporcionada pelo trilho. 

Na contexto da aplicação proposta foi concebido duas possibilidades para a
fixação do sistema de trilhos. A primeira solução consiste em um sistema
semelhante ao Roboturb, apresentado na seção \ref{sec::rail}. O sistema proposto
se trata de um manipulador robótico, com fixação diretamente na própria pá da
turbina. \textbf{O tamanho do manipulador tem relação direta com o tamanho e
peso do mecanismo de metalização. Outro fator determinante para a viabilidade da
solução é o espaço de trabalho disponível e a distância que o efetuador deve
manter da superfície da pá.}

O trilho deve ser flexível para ser capaz de acompanhar a curvatura da pá e
possibilitar diversas opções de posicionamento. Imãs permanentes podem ser
utilizados como solução de fixação. entrentanto \textbf{deve-se verificar a
resistência do material da superfície da pá, pois dependendo da força magnética
necessária para aguentar o peso do sistema é possível que o processo de
acoplamento e retirada dos imãs danifique a superfície}. A utilização de
ventosas passivas também pode ser adotada, porém devido a própria natureza desse
sistema é necessário que se desenvolva um sistema de segurança para a detecção
de uma possível perda de poder de sucção. Aceleromêtros podem ser empregados
para verificar deslocamentos e desativar completamente o processo de
metalização. Uma abordagem de segurança mais ativa pode consistir no estudo do
emprego de ventosas ativas para eliminar a chance de que o robô caia.

Para a solução baseada na fixação diretamente na pá, é necessário, que
\textbf{após o processo de metalização, seja possível a fixação em um ponto
recém processado. Caso haja um tempo elevado de resfriamento ou fixação do
material, essa solução se torna impossibilitada uma vez que para a total
cobertura da superfície da pá é necessário que a posição do trilho seja alterada
pelo menos uma vez.}


 \begin{itemize}
   \item Falar sobre as vantagens da seção de robos de trilhos
 \end{itemize}
 Gerais	
 Trilho na pá
  \item pensar metodo de instalação para partes altas da pá
\end{itemize}

Trilho preso à turbina
\begin{itemize}
  \item pensar método de fixação ao rotor e chão. Magnetico talvez, pressão,
  trilho extensível.
  \item nariz gira junto?
  \item vantagem do trilho ser reto e movimento ser em 1D.
  \item ponderar facilidade de instalacao
  \item verificar qual o alcance que o manipulador deverá ter
  
\end{itemize}