\subsection{Projeto de robôs escaladores}\label{proj_climbers}

Nesta subseção, consideram-se soluções para HVOF de pás de turbinas robôs
escaladores com fusão das tecnologias documentadas na
seção~\ref{sota}, subseção~\ref{sota_climbers}. Duas soluções serão abordadas:
a primeira será uma versão adaptada do robô \emph{The Climber}, ICM, e a
seguinte a fusão das tecnologia do Roboturb e o \emph{Climbing robot for Grit
Blasting}.

\textbf{Versão adaptada The Climber}  

O robô \emph{The Climber}, ICM, é uma solução comercial que atende muitas das
especificações HVOF e possibilita aperfeiçoamento sem comprometer sua
estrutura. O robô possui sistema de adesão por sucção e locomoção através de
esteiras flexíveis. O sistema já foi testado em ambientes de alta
periculosidade, como turbinas eólicas, usinas hidrelétricas e outros, porém
deve-se ter conhecimento preciso das \textbf{características do ambiente,
compartimento da turbina bulbo, como temperatura e umidade}. Podemos dividir o
projeto em quatro sistemas: locomoção, adesão, manipulador e autonomia.

O sistema desenvolvido em \cite{kim2008development} têm mecanismos de
locomoção por esteiras e adesão por sucção compostos por polias, correias de
borracha, ventosas, válvulas para cada ventosa, motores DC para as polias,
sistemas de controle para as válvulas e para os motores . \emph{The Climber}
utiliza apenas uma câmara de vácuo, em vez de ventosas, e esteiras flexíveis que
permitem maior suavidade e continuidade ao movimento. A solução por uma única
câmara parece mais vantajosa, já que o robô consegue se locomover em curvaturas
de até 30 cm de raio.

No caso específico da aplicação HVOF, o processo é realizado com
manipulador enquanto o robô percorre a pá da turbina. A locomoção do
robô sob a pá levanta algumas questões de projeto: \textbf{a
temperatura da turbina durante o procedimento pode inviabilizar a solução
devido ao contato contínuo robô/turbina}; deve ser considerado o \textbf{tempo
de cura do HVOF para a locomoção sob a pá}; verificar se há \textbf{acesso a
todos os pontos da pá, limites em relação à parede, disposição da pá,
sobreposição entre pás}; e \textbf{como se comporta o robô em curvaturas
acentuadas, e o quanto a pá pode ser girada para facilitar locomoção}. A
viabilidade de qualquer projeto com robôs escaladores dependem dessas questões,
independente da tecnologia de locomoção e adesão.

O sistema de adesão ativo por sucção deve considerar a \textbf{força máxima que
pode ser realizada sem danificar o revestimento ou a superfície da pá} e
\textbf{o peso total do conjunto do sistema HVOF}. Além disso, um mecanismo
inteligente de segurança, possivelmente utilizando acelerômetros e outros
sensores, deve ser embarcado no robô para garantir o \textbf{desligamento do
sistema (tempo a ser estimado)} em caso de queda ou falha, dependendo das
\textbf{condições de segurança}. A solução de um robô com planejamento de
trajetória aumenta a segurança da operação e o controle ótimo do mecanismo de
adesão pode limitar a força máxima de sucção.

O manipulador a ser projetado para aplicação HVOF deve possuir as seguintes
características: ser leve para não comprometer a adesão e equilíbrio do sistema
móvel; \textbf{rápido e preciso conforme requer a aplicação HVOF}; modular, já
que a operação será realizada in-situ e a sua montagem deverá ser realizada
caso o \textbf{acesso ao local seja incompatível com as dimensões do robô com
manipulador}; não é necessário possuir grandes dimensões, pois o robô é móvel e
pode percorrer a pá, porém deve ser suficiente para operar em \textbf{pontos de
difícil acesso à base} e considerar a \textbf{distância mínima entre pistola
HVOF e pá}; e ser capaz de sustentar a força gerada pela pistola HVOF. A
literatura sobre manipuladores é bastante consolidada, sendo muitos dos
problemas citados já resolvidos e disponíveis no mercado, como o desenvolvido
em \cite{manzdevelopment}.

O manipulador também executará a função de remoção de material, preparação para
HVOF e medida de integridade da pá. Há diversos meios para remoção do
revestimento, como laser, abrasão e outros. O manipulador poderá ser modificado
de acordo com o procedimento (remoção ou aplicação do revestimento) ou um outro
sistema pode ser utilizado, dependendo de \textbf{como é realizado o
procedimento de remoção}. A tecnologia que verifica a necessidade de
revestimento, com sensores laser e ultrassom, e poderá indicar o \textbf{mapa
ou apenas realizar um teste de sucesso/falha} \citep{escaler2006detection}.

O sistema autônomo de um robô móvel é a inteligência do robô. Ele abrange o
controle de missão, ou seja, o planejamento e execução das tarefas.
A locomoção será realizada pelo controle dos motores em conjunto com o controle do
sistema ativo de adesão por sucção, o planejamento de trajetória, desvio de
obstáculos e mapeamento do ambiente, através de um conjunto de sensores, como
laser e acelerômetros. O controle do manipulador poderá ser cinemático por
servovisão ou pela estruturação do ambiente. E um sistema de suporte do veículo
icará responsável pela segurança, bom funcionamento e gerenciamento de potência
do robô.

As características descritas acima como solução de um robô escalador impede a
troca automática entre pás. Um robô escalador com tecnologia de avanço pendurado
por braços é uma solução muito custosa em termos de controle e estrutura
mecânica. Outra solução seria um robô com locomoção por segmentos deslizantes,
como o RRX3, e adesão por sucção, porém a flexibilidade exigida para a locomoção
entre pás e a distância entre turbinas complexifica o projeto. Dessa forma, a
troca entre pás deverá ser manual e, caso o \textbf{cone da turbina também seja
revestido por HVOF}, o operador deve manualmente transportar o robô.

\textbf{Versão adaptada Roboturb}

O Roboturb, como já descrito na subseção~\ref{sec::rail}, é um
manipulador que se locomove em um trilho, este acoplado à pá da turbina
por ventosas (sucção). A solução não permite a extensão
do manipulador, já que o peso desequilibra a estrutura e não há torque para
compensar a força exercida no efetuador durante a operação HVOF. A segunda
solução de robôs escaladores é adicionar um trilho perpendicular e transformar
o Roboturb em um robô móvel, com locomoção através de dois trilhos, idéia
semelhante ao \emph{Climbing robot for Grit Blasting}, que utiliza duas
plataformas deslizantes com ventosas.

Os trilhos são compostos por esteiras flexíveis nas extremidades para a
locomoção, como \emph{The Climber}, e as ventosas são ativas e distribuídas por
todo o trilho. O manipulador só necessitaria mover em um dos trilhos para
percorrer toda a pá, já que os trilhos também se movimentam. 

A solução de trilhos móveis com manipulador é dependente à curvatura da pá da
turbina e o aumento da flexibilidade do trilho para se locomover sob a pá pode
impedir a movimentação do manipulador. Dessa forma, é considerada uma solução
muito específica e restrita à aplicação.