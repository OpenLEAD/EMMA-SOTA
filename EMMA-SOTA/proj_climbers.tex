\subsection{Projeto de robôs escaladores}\label{proj_climbers}

Nesta subseção, consideram-se soluções para HVOF de pás de turbinas robôs
escaladores com fusão das tecnologias documentadas na
seção~\ref{sota}, subseção~\ref{sota_climbers}.

A solução mais completa, que mais atende às especificações HVOF e possibilita
aperfeiçoamento sem comprometer sua estrutura, é o robô \emph{The Climber}, ICM
Robotics. O robô possui sistema de adesão por sucção e locomoção através de
esteiras. O sistema já foi testado em ambientes de alta periculosidade, como
turbinas eólicas, usinas hidrelétricas e outros, porém deve-se ter conhecimento
preciso das \textbf{características do ambiente, como temperatura e umidade}.
Podemos dividir o projeto em quatro sistemas: locomoção, adesão, manipulador e
autonomia.

No caso específico da aplicação HVOF, o processo é realizado com
manipulador enquanto o robô percorre a pá da turbina. A locomoção do
robô por esteiras levanta algumas questões de projeto: \textbf{a temperatura da
turbina durante o procedimento pode inviabilizar a solução devido ao contato
contínuo robô/turbina}; deve ser considerado o \textbf{tempo de cura do HVOF
para a locomoção sob a pá}; verificar \textbf{o acesso do robô a todos os pontos
da pá, seus limites em relação à parede, disposição, sobreposição entre pás}; e
\textbf{a mobilidade do robô em curvaturas acentuadas, e o quanto a pá pode ser
girada para facilitar locomoção}.

O sistema de adesão ativo por sucção deve considerar a \textbf{força máxima que
pode ser realizada sem danificar o revestimento ou a superfície da pá} e
\textbf{o peso total do conjunto do sistema HVOF}. Além disso, um mecanismo
inteligente de segurança, possivelmente utilizando acelerômetros e outros
sensores, deve ser embarcado no robô para garantir o \textbf{desligamento do
sistema (tempo a ser estimado)} em caso de queda ou falha, dependendo das
\textbf{condições de segurança}.

O manipulador a ser projetado para aplicação HVOF deve possuir as seguintes
características: ser leve para não comprometer a adesão e equilíbrio do sistema
móvel; \textbf{rápido e preciso conforme requer a aplicação HVOF}; modular, já que a operação será realizada
in-situ e a sua montagem deverá ser realizada caso o \textbf{acesso ao local seja
incompatível com as dimensões do robô com manipulador}; não é necessário
possuir grandes dimensões, pois o robô é móvel e pode percorrer a pá,
porém deve ser suficiente para operar em \textbf{pontos de difícil acesso à
base} e considerar a \textbf{distância mínima entre pistola HVOF e pá}; e ser
capaz de sustentar a força gerada pela pistola HVOF. A literatura
sobre manipuladores é bastante consolidada, sendo muitos dos problemas citados
já resolvidos e disponíveis no mercado, como o desenvolvido em
\cite{manzdevelopment}.

O manipulador também executará a função de remoção de material, preparação para
HVOF e medida de integridade da pá. Há diversos meios para remoção do
revestimento, como laser, abrasão e outros. O manipulador poderá ser modificado
de acordo com o procedimento (remoção ou aplicação do revestimento) ou um outro
sistema pode ser utilizado, dependendo de \textbf{como é realizado o procedimento de remoção}. A tecnologia que
verifica a necessidade de revestimento utiliza sensores laser e poderá indicar o
mapa ou apenas realizar um teste de sucesso/falha.
%referencia aqui

O sistema autônomo de um robô móvel é a inteligência do robô. Ele abrange a
locomoção automática sob a pá e a execução das tarefas. A locomoção será
realizada pelo controle dos motores das esteiras em conjunto com o controle do
sistema ativo de adesão por sucção, planejamento de trajetória, desvio de
obstáculos e mapeamento do ambiente, através de um conjunto de sensores, como
laser e acelerômetros. O controle do manipulador poderá ser cinemático por
servovisão ou pela estruturação do ambiente.
% referencias aqui

As características do robô impedem a troca automática entre pás. Um robô
escalador com tecnologia de avanço pendurado por braços é uma solução muito
custosa em termos de controle e estrutura mecânica. Outra solução seria um robô
com locomoção por segmentos deslizantes, como o RRX3, e adesão por sucção, porém
a flexibilidade exigida para a locomoção entre pás e a distância inviabiliza o
projeto. Dessa forma, a troca entre pás deverá ser manual e, caso o \textbf{cone
da turbina seja revestido}, o operador deve manualmente transportar o robô.

% TODO Rail em movimento