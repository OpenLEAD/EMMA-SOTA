\subsection{Construção do ambiente e projeto de bases
mecânicas}
Os estudos das possíveis soluções exigiu uma visualização mais detalhada do
volume livre no interior da turbina. Para isso, foi recriado o ambiente da
turbina em CAD 3D, a partir dos desenhos 2D fornecidos pelo cliente. O modelo
tridimensional do aro câmara permite o estudo e o dimensionamento geométrico de
alcance do manipulador para cada solução. Não foram necessários
detalhamentos de todos os componentes, podendo ser apenas considerados, e
representados com maior precisão, os perfis externos do túnel à montante, o
estator, o corpo da turbina, as palhetas e uma pequena região à jusante, além dos acessos por escotilha superior e inferior.
A figura~\ref{ambiente3d} apresenta o ambiente da turbina em CAD e os possíveis
acessos para realização das intervenções.

A solução pela escotilha superior, devido ao espaço reduzido de entrada, não
permite a utilização de manipuladores de grande porte, sendo escolhido o KUKA
LBR 820. Este manipulador não possui alcance para realizar a operação em toda a
pá de uma só vez, o que exige uma base customizada que permita o
posicionamento do manipulador para realização das operações por etapas. Para isso, foi estudada
uma estrutura de base que permitisse diferentes
posicionamentos para o manipulador no interior do aro câmara, de forma que este
pudesse cobrir toda a superfície da pá. A base consiste em 3 atuadores
telescópicos que permitem a extensão do sistema para prover o alcance
necessário ao manipulador e o recolhimento para uma configuração incial que
permite a entrada do manipulador no aro câmara, sem o risco de choques ou
interferências indesejadas. Além disso, uma junta rotativa oferece mais um grau
de liberdade para o sistema, facilitando o acesso do manipulador à toda a
superfície da pá. A base é composta por cilindros de diâmetro maximizado, o
que, devido ao momento de inércia elevado, oferece grande rigidez à flexão,
minimizando erros de posicionamento e vibração excessiva. A
figura~\ref{base_superior_conceito} apresenta o conceito da base do manipulador em duas configurações: recolhida (configuração de entrada) e extendida (configuração de operação).
A figura~\ref{base_superior_interior} apresenta a base com o manipulador no
interior do aro câmara na configuração de entrada e em uma das configurações de
operação.




%\begin{figure}[h!]
%\centering
	%\includegraphics[width=\columnwidth]{figs/estudo/solid/arquivo.pdf} 
	%\caption{Trilho fora da pá com processamento em diagonal.}
	%\label{rail2}
%\end{figure}