\subsubsection{Projetos com manipuladores industriais fixos}\label{proj_manip}
Há diversos manipuladores robóticos industriais com as especificações
necessárias para a realização da tarefa de metalização por HVOF. As empresas
Fanuc, Motoman, ABB e KUKA fabricam manipuladores com dimensões compatíveis com o
acesso pela escotilha inferior e velocidade, precisão, e espaço de trabalho que
cumprem os requisitos para a execução do processo em todo um lado da pá, em uma
base fixa. Há apenas a necessidade de escolher a posição correta do manipulador
em relação à pá, maximizando a sua área de trabalho. Como as pás podem ser
giradas até um ângulo de $14.5^o$, são discutidas as ideias de posicionamento
do manipulador entre as pás, a fim de executar a operação em ambos os lados da
pá (um lado de cada pá), e o posicionamento fixo à frente da pá.

\textbf{Posicionamento entre pás}
A figura~\ref{fig::andaime} mostra o espaço entre as pás da turbina, dentro do
aro câmara. Um robô manipulador de médio porte pode ser fixado em uma base
magnética, na posição que se encontra a escada da figura~\ref{fig::andaime}. 
