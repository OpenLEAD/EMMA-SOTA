\subsection{Acesso pela escotilha de dimensão pequena}
%TODO Elael: prós e contras do acesso, soluções: manipuladores
% industriais, manipuladores customizados, trilhos. Incluir figuras do
% posicionamento das pás e cálculos do tamanho mínimo do manipulador

Essa escotilha localizada no topo do aro câmara possui uma abertura de apenas
35cm de diâmetro, o que cria um desafio quando se pensa em utiliza-la como ponto
de acesso para um robô. Por outro lado sua proximidade às pás e a área livre fora do aro
câmara criam possibilidades interessantes para seu uso.

\textbf{Vantagens}
\begin{itemize}
  \item Estabilidade da fixação do robô
  \item Ponto de referência
  \item Pórtico rolante para posicionar o robô na escotilha
\end{itemize}

\textbf{Desvantagens}
\begin{itemize}
  \item Dificuldade de encontrar robôs de tal dimensão (35cm diâmetro)
  \item Necessidade de retirar e inserir o robô quando rodar as pás
  \item Solução focada na UHE Jirau
\end{itemize}

A escolha do robô está primariamente associada ao seu alcance, e o alcance, por
sua vez, a posição do robô com relação à pá. Por outro lado, apenas uma pequena
parcela dos robôs comerciais possuem a dimensão necessária para atravessar a
escotilha. Sendo assim, o estudo foi direcionada para o uso do KUKA Light Weight
(LBR iiwa 14 R820), robô cuja diagonal da base é inferior aos 35 cm da
escotilha.

O LBR R820 pesa 30kg, possui 7 eixos e suporta um \textit{payload} de 14kg,
suficiente por uma pequena margem para carregar o equipamento de
\textit{coating}. Entretanto, é necessário um estudo aprofundado para valida-lo
quanto aos requisitos de velocidade e precisão quando percorrendo a trajetória
de \textit{coating}.

Tendo como objetivo posicionar o LBR R820 em uma posição onde seja capaz de
trabalhar sobre toda a pá, um modelo de base articulada foi proposto. A base
composta de dois elos interligados por uma junta de rotação é fixada sobre a
própria escotilha de um lado, o que lhe dá um ponto de referência e fixação, e
do outro possui o braço robótico. Para que seja possível cobrir toda a pá,
a base deve ser inserida em possivelmente diversas angulações com respeito ao
eixo de inserção, tendo o intúito de alterar o eixo de rotação da junta que
conecta os dois segmentos da base. Essa junta também precisa ser sua posição
alterada, o que pode ser realizado manualmente, sem a necessidade de um motor
para atua-la.

Intrudizir o robô, composto pelo conjunto base-LBR R820, é uma tarefa cuidadosa
pois a extensão total será maior que a distância do topo do aro câmara ao
\emph{nose} da turbina. Ou seja, o braço e a base precisarão ser rotacionados
durante o processo de inserção, o que acarretará no desalinhamento do centro de
massa (com relação ao eixo perpendicular a escotilha) e exigirá uma guia para
resistir ao torque gerado por esse desalinhamento.

Com o robô fixado na escotilha e a junta da base travada, são esperados torques
inferiores a 3000Nm sobre a junta e 4000Nm sobre a base, durante a operação de
coating com o robô. Apesar de tido como valor máximo, os torques, para
definições do tamanho do braço longe dos valores extremos, devem estar em torno
de metade desse limite.

Ao se avaliar as possibilidades de realizar o coating sem a necessidade de
placas de sacrifício, percebe-se que isso poderá ser realizado com a junta da
base paralela à pá (que para efeito de cálculo foi assumida com uma angulação de 
$45^o$ com relação ao eixo da turbina). Para esse caso a junta precisa realizar
uma rotação à velcidade mínima de 34 rad/min. Essa velocidade possibilita a
pistola de coating percorrer verticalmente a pá sem pausa. Percorrer
horizontalmente a pá é impossível pelas restrições nas dimensções da base. Para
determinar essas condições, foram utilizadas as informações reais de extenção da
pá e área de trabalho do robô, entretanto foi desprezada a curvatura da pá.

