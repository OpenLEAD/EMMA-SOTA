\section{Descrição do problema}\label{sec:consideracoes}
%TODO Renan: introdução do problema
%TODO Renan: Descrição do processo HVOF (questoes HVOF) e cavitação
%TODO Gabriel: Caso JIRAU/contextualização (Ambiente). Detalhe dos acessos 
%TODO Elael: Tarefas do Robô
O fenômeno de cavitação em hidroturbinas provoca redução da eficiência na
geração de energia e desgaste superficial por erosão. Uma solução preventiva é
o revestimento por metalização das pás, o qual aumenta a eficiência na
geração de energia por gerar uma estrutura mais lamelar, e fornecer maior
resistência a desgastes abrasivos, corrosivos e erosivos.

A cavitação é a formação de cavidades de vapor, bolhas, em um líquido devido a
quedas repentinas de pressão. Quando o mesmo líquido é sujeito a aumento de
pressão, as bolhas implodem Em hidroturbinas, há perda de pressão na água quando
ocorre escoamento, provocando vaporização e,


Nestas condições, caso a mistura atingir alguma região aonde a pressão absoluta for
novamente superior à pressão de vapor, haverá o colapso das bolhas com retorno à fase líquida.
Entretanto, como o volume específico do líquido é inferior ao volume específico do vapor, o colapso
das bolhas implicará a existência de um vazio, proporcionando o aparecimento de ondas de choque.
Este processo de formação, crescimento e colapso das bolhas de vapor no meio líquido é
chamado de cavitação. A cavitação pode ocorrer em qualquer líquido no qual a pressão estática local
do fluido esteja igual ou abaixo da pressão de vapor desta substância, sem alteração da temperatura.
Com a formação das bolhas de vapor há uma mudança nas características do escoamento, que
pode tornar-se transiente. Esta mudança pode ocasionar oscilações no escoamento e vibrações na
máquina que por conseqüência pode afetar o rendimento do sistema hidráulico. Como dissemos
anteriormente, com o colapso das bolhas ocorre um micro-jato d’água e ondas de choque, e se isso
acontecer próximo a uma superfície, pode vir a ocorrer desgaste por erosão. A cavitação em
máquinas hidráulicas apresenta o surgimento de efeitos indesejados como: instabilidade no
escoamento do fluido, vibração, ruído excessivo e desgaste superficial nas paredes das superfícies
metálicas. 

