\subsection{Descrição das tarefas do robô}
%TODO Elael: Tarefas do Robô
Esta subseção descreve as tarefas básicas do robô para o revestimento de
turbinas \textit{in situ}. Em linhas gerais, o robô a ser desenvolvido deve ser
capaz de realizar a tarefa de revestimento tal qual seria feita caso a pá não estivesse instalada na
tubina e de uma maneira autônoma. A pá, antes de ser submetida ao
processo de revestimento, deve estar em conformidade com o gabarito, perfil hidráulico de uma pá
intacta. Portanto, uma tarefa do robô é realizar o mapeamento do perfil
hidráulico, construir um modelo 3D e analisar imperfeições.

Em caso de deformações, causados por cavitação e abrasão, estas precisam
ser removidas manualmente ou de forma automatizada, possivelmente por
soldagem, a qual pode pode ser uma tarefa do robô. A tarefa de soldagem pode
ser realizada por operador, manualmente, por não possuir todas as restrições
da tarefa de revestimento (velocidade, precisão, carga e etc), porém o ambiente
pode dificultar a operação de forma que a execução por um robô seja
indispensável. 

Após as pás estarem de acordo com o gabarito, faz-se a
identificação do desgaste do revestimento, medindo sua espessura em um
número finito de pontos sobre a superfície da pá. Manualmente esse
processo é realizado eficientemente em 10 min, justificando a não necessidade de
esta ser uma tarefa do robô. Porém, em caso de necessidade de aplicação
de novo revestimento, é necessária a remoção do revestimento antigo por
jateamento, a fim de deixar a superfície rugosa e aumentar sua aderência. A
tarefa de jateamento pode ser realizada pelo robô, ou manualmente
com ajuda de andaimes internos ao ambiente do aro câmara. Como ambos os lados da pá são revestidos, o
jateamento deve ser realizado em ambos os lados. Vale ressaltar que, em teoria,
pode-se aplicar revestimento por metalização sem retirar o último revestimento, porém não é o
recomendado.
%Segue-se o exemplo de empresas de aviação, onde existe a
%prática de retirar todo o revestimento antigo antes de aplicar o novo.

Por fim, o robô deverá aplicar o revestimento como
forma de prevenir o dano causado pelos fenômenos abrasivos. O robô projetado
para fazer o revestimento precisa preencher todos os requisitos discutidos na
subseção~\ref{sec::desc_hvof} e ser adaptável ao ambiente, cujos os requisitos
são discutidos na subseção~\ref{sec::desc_contex}.

Das tarefas a serem relizadas, são destacadas as seguintes:
Tarefas que podem ser executadas manualmente:
\begin{itemize}
  \item Inspeção e análise de danos na pá, tanto para reparo quanto para
  revestimento.
  \item Reparo.
  \item Montagem do sistema.
  \item Jateamento da superfície.
\end{itemize}

Tarefas que poderão ser executadas pelo robô:
\begin{itemize}
  \item Modelar o perfil hidráulico.
  \item Calibração.
  \item Jateamento.
  \item Reparo (soldagem e esmerilhamento).
  \item Revestimento por metalização.
\end{itemize}
