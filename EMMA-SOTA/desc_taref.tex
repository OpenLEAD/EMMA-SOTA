\subsection{Descrição das tarefas do robô}
%TODO Elael: Tarefas do Robô


EMMA SOTA: SOTA de aplicação HVOF

.Processo atual de Hard coating
.Processo de cavitação
.Requisitos do ambiente

Tarefas do Robô:
1 - Robô deverá modelar perfil hidráulico da pá. Existe erosão e cavitação e
esses fenômenos são influenciados pela variação de queda d'água. Estudos
mostraram que com diferenças maiores de 12 metros, os problemas de cavitação são
minimizados. Há danos em ambos os lados da pá devido a presença de
fenômenos de alta e baixa pressão.
2 - Há necessidade de fazer remoção e jateamento para deixar a superfície rugosa
e aumentar a aderência (óxido de alumínio). Esta tarefa não precisa ser
realizada pelo robô. Existe a possibilidade de montar um andaime.
3 - Sim. Ambos os lados da pá são revestidos. O jateamento e preparação devem
ser realizados em ambos os lados da pá.
4 = 2.
5 - As operações também deverão ser realizadas no
cone da turbina.

Não. Apenas na superfície da pá.
6 - A preparação não é automatizada. Mas é necessária a inspeção e análise de
cavitação para realizar de forma automática o reparo. A aplicação de HVOF tem
como premissa que os problemas relacionados à cavitação foram corrigidos e o
peril hidráulico da pá está em conformidade com o gabarito.
7 - ? Extra: - Medir hard coating é simples e rápido. Pontos são medidos. São 10
minutos por pá. Não será realizado pelo robô.
- Pode-se aplicar hard coating sem retirar o ultimo revestimento. Não é o
melhor, mas possível. A TAP (aviação) retira tudo antes de aplicar. São diversas
camadas finas.


Extras:
- Não se sabe a força no eixo Z. Cálculo é necessário ou experimento em célula
de carga.
- Todo o material injetado tem perda de 40%.
- A pistola não pode permanecer na pá por muito tempo, pois acúmulo de material
é criado. Troca de direção ou sentido na movimentação do manipulador é
considerado como parada. As trocas deverão ser realizadas em áreas exteriores à
superfície da pá. Entretanto, chapas de sacrifício podem ser colocadas para o
processamento de pequenas áreas da pá. Fitas de sacrifício não são eficientes
ainda.
- Espessura de coating está relacionada à velocidade e quantidade de vezes que a
pistola passa.
- Tempo médio de processamento atual é de aprox 1 dia por superfície da pá. 6
horas um lado.
- Menor escotilha de acesso da usina tem 80 cm.
- Robô é selado devido ao pó.

Solução: Robô com braço curto e divisão da pá em xadrez. Elevação do robô em relação à pá. Atuadores hidráulicos?
