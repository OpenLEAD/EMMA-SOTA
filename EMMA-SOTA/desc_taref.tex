\subsection{Descrição das tarefas do robô}
%TODO Elael: Tarefas do Robô

Em linhas gerais, o robô a ser desenvolvido deve ser capaz de realizar a tarefa
de revestimento tal qual seria feita caso a pá não estivesse instalada na
tubina. Porém, além das dificuldades relacionadas ao acesso e à geometria da
região de trabalho dentro da unidade geradora, as pás podem conter danos
causados pela erosão e cavitação que devem ser corrigidos antes do processo de
revestimento.

Toda a pá, antes de ser submetida ao processo de revestimento, deve estar
em conformidade com o gabarito, perfil hidráulico de uma pá
intacta. Portanto, uma tarefa do robô é realizar o mapeamento do perfil
hidráulico e construi um modelo 3D.

Na usina de Jirau, as pás sofrem erosão e cavitação, onde esses fenômenos são
influenciados pela variação de queda d'água. Estudos mostraram que com
diferenças maiores de 12 metros, os problemas de cavitação são minimizados,
porém a hidrelétrica está frequentemente em operação abaixo desse nível.
Dessa forma, há danos em ambos os lados da pá devido a presença de fenômenos de
alta e baixa pressão. Essas deformações precisam ser removidas manualmente ou de
forma automatizada, possivelmente por soldagem, mas é necessária a inspeção e
análise de cavitação para realizar de forma automática o reparo. O
\textit{hardcoating} vem como forma de prevenir o dano causa por tais fenômenos,
porém este também sofre desgaste. Fortuitamente, indetificar o dano sofrido pelo
revestimento é simples e rápido. A espessura do \textit{coating} é medida em um
número finito de pontos sobre a superfície da pá, no total, leva-se em torno de
10 minutos por pá. E essa tarefa não será realizado pelo robô.

Após as pás estarem de acordo com o gabarito, ainda há necessidade de fazer
remoção e jateamento para deixar a superfície rugosa e aumentar a aderência do
revestimento (composto de óxido de alumínio). Esta tarefa não precisa ser
realizada pelo robô, existe a possibilidade de montar um andaime. Como ambos os
lados da pá são revestidos. O jateamento e preparação devem ser realizados em
ambos os lados da pá, porém cone da turbina não recebe esse tratamento. Em
teoria, pode-se aplicar \textit{hardcoating} sem retirar o último revestimento,
porém não é o recomendado. Segue-se o exemplo de empresas de aviação, onde
existe a prática de retirar todo o revestimento antigo antes de aplicar o novo.

O robô desenhado para fazer o coating precisa ser capaz de manter sua precisão
de movimento enquanto resiste a forção imposta pelo empuxo de saída do gás e do
material da pistola de metalização. Tal força, exercida na direção contrária da
direção de aplicação, não há estimativa. Cálculo é necessário ou experimento em
célula de carga para definir qual a grandeza desta. O robô precisa de capaz de
imprimir sobre a pistola uma trajetória que não envolva troca de direções sobre
a pá, pois a chama não pode permanecer na pá por muito tempo, caso contrário
ocorre acúmulo de material sobre o ponto. Troca de direção ou sentido na
movimentação do manipulador é considerado como parada. Logo, as trocas deverão
ser realizadas em áreas exteriores à superfície da pá. Entretanto, \emph{Chapas de
Sacrifício} podem ser colocadas para o processamento de pequenas áreas da pá. Que
são apenas chapas metálicas colocadas como entreposto para evitar que o material
seja depositado em determinada região, coberta pela chapa. Existe um mecanismo
semelhante, conhecido por Fitas de Sacrifício, porém, apesar de teoricamente
mais prático, não são eficientes ainda.

Espessura do \textit{hardcoating} está relacionada à velocidade e quantidade de vezes que
a pistola passa. O que torna esse um processo naturalmente demorado. O tempo
médio atual de processamento é de aproximadamente um dia por superfície da pá,
levando seis horas em cada face. Tendo então o robô que ser projetado para
funcionar continuamente sob as condições ambientais locais, ou ter paradas
planejadas, em ambos os casos a londa duração da tarefa precisa ser levada em
consideração. O \textit{hardcoating} é um processo não apenas dispendioso em
tempo, como também é material. Todo o material injetado tem perda de 40\%, e
esse material perdido se dispersa sobre o ambiente como um pó de uma
granularidade diminuta. Esse pó impõe ao robô mais uma restrição, pois ele deve
ser selado como medida para evitar danos ao sistema. 

Das tarefas a serem relizadas, são destacas as seguintes.
Tarefas necessárias, porém que não precisam ser exercidas pelo robô:
\begin{itemize}
  \item Modelar perfil hidráulico da pá, contruir gabarito.
  \item Inspeção e análise de cavitação.
  \item Jateamento da superfície.
\end{itemize}

Tarefas, ou restrições, que serão executadas pelo robô:
\begin{itemize}
  \item Estabilizar a pistola.
  \item Realizar trajeto com a pistola sem troca de direção ou sentido sobre a
  pá.
  \item Levar em conta a longa duração da tarefa.
  \item Ser selado ao excesso de pó gerado pelo processo de \textit{coating}.
\end{itemize}
