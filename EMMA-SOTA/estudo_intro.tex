O estudo de viabilidade consiste em avaliar as soluções conceituas mais simples 
da seção~\ref{sec:projeto}, ou seja, um estudo técnico específico para as
seguintes soluções: acesso pela escotilha superior com manipulador industrial de
pequeno porte e base customizada operada eletronicamente; acesso pela escotilha
inferior com manipulador industrial de médio porte e base fixa magnética; e
acesso pela jusante com manipulador industrial de grande porte e base fixa
magnética.

A primeira etapa do estudo de viabilidade consiste em pesquisa de mercado por
manipuladores industriais, levando em consideração os requisitos do processo de
metalização (velocidade e payload), espaço de trabalho, e as dimensões e peso
para compatibilidade com o acesso. O resultado da pesquisa mostrou que, em
relação ao acesso pela escotilha superior, as dimensões reduzidas restringiram muito a busca e apenas o
manipulador LBR 820 da Kuka satisfaz aos requisitos. Para os outros acessos, há
variadas soluções de manipuladores industriais.

As outras etapas são avaliações técnicas e são divididas nas seguintes
subseções: estudo geométrico para confirmar alcance do manipulador em todos os
pontos da pá; construção do ambiente 3D em SolidWorks, projeto de bases
mecânicas, movimentação e logística de acesso; simulação do espaço de trabalho e
e estudo de manipulabilidade; estudo de placas de sacrifício, mecanismos atuados
para interromper o revestimento e materiais. 
