\subsubsection{Projeto de robôs em trilhos}\label{proj_rail}
 % attach a rail to the blade and move it manually
 
 % attach a rail one the nose and ground, 1D movement and move the blade to
A utlização de um manipulador robótico sobre trilhos satisfaz todos os
requisitos para a realização de um processo de inspeção e metalização utilizando a técnica HVOF. O desenvolvimento
de um sistema compacto para o transporte através do acesso pela escotilha
inferior e sua instalação no aro câmara da turbina são possíveis, pois as
dimensões do manipulador podem ser reduzidas por meio da
mobilidade extra proporcionada pela introdução do trilho.

No contexto da aplicação proposta, foram concebidas duas possibilidades para a
fixação do sistema de trilhos. A primeira solução consiste em um sistema
semelhante ao Roboturb, apresentado na seção \ref{sec::rail}. O sistema proposto
se trata de um manipulador robótico com fixação diretamente na pá da
turbina. O trilho deverá ser flexível para ser capaz de acompanhar a curvatura
da pá e possibilitar diversas opções de posicionamento. Como o material da pá
não possui alta permeabilidade magnética (Inox 420), a solução de fixação seria
por ventosas ativas e com material específico para suportar as grandes
variações de temperatura que a pá pode alcançar (temperatura ambiente a
$100^oC$ durante a metalização).

Uma abrangente pesquisa de robôs comerciais industriais de pequeno porte apontou
que há manipuladores com carga entre 12 e 20 kg e velocidade necessários,
sendo o LBR da Kuka o que possui melhor benefício peso/alcance, 30 Kg e 820 mm,
respectivamente. %Para este manipulador, a metalização deverá ser
%realizada em, pelo menos, quatro etapas com quatro trilhos diferentes e
%customizados, e placas de sacrifício para evitar mau aplicação da metalização
%durante as trocas de sentido na movimentação do robô.

A fixação de um trilho na pá apresenta diversas complexidades, como: a
necessidade de manualmente instalar/desinstalar o sistema trilho/robô diversas
vezes em cada pá; o projeto do trilho customizado e flexível; e ventosas ativas
especiais que suportam variação de temperatura.

A alternativa para se evitar o contato com a pá consiste em um único trilho
retilíneo fixado por bases magnéticas ou solda, no solo do aro câmara. Como o
robô não possui alcance de toda a pá, há, ainda, a necessidade de posições
verticais diferentes. A pá pode ser processada em movimentos circulares ou
lineares e, em ambos os casos, o manipulador ficará responsável pela velocidade,
posição e orientação do processo. A troca de sentido de movimento deverá
ocorrer fora da pá ou devem ser utilizadas placas de sacrifício. A

Esse tipo de abordagem simplifica a movimentação do robô no
trilho, uma vez que o trilho seria totalmente reto, e possibilitaria a
metalização de um dos lados das quatro pás com uma única instalação de base.
Porém, mesmo nesta solução, a altura do trilho deverá ser ajustada três vezes para
cada lado de pá.

Em ambos os sistemas propostos, é necessária a implementação de um sistema de
localização do robô em relação à pá, tornando possível a geração de um
planejamento de trajetórias para o processo de metalização. O sistema de
localização pode ser concebido por sensores externos
ao robô (câmeras e outros), ou instalados no próprio manipulador/base.

\textbf{Conclusão da solução por robôs em trilhos}

A solução com trilho externo se mostrou vantajosa em comparação ao robô em
trilho customizado acoplado à pá, devido à complexidade e intervenções
manuais. Há a possibilidade de utilizar um manipulador industrial, tornando o
foco do projeto em processamento de sinais, mapeamento, localização e controle,
além da construção do trilho. Porém, a montagem da estrutura e a instalação de
todo o sistema atrás da pá podem ser custosas, sendo esta ainda uma solução
considerada complexa.