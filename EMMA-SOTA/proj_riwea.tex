% RIWEA-like rail on which the coating system
%  place blade horizontal and use wheels, against the blade's rim,to move a rail
  % along the blade's shape
  %  two-arm solution (or one arm + magnetic attachment point) to allow reaching
  % the top of the blade from the very bottom
\subsection{Robô Bipartido}

Esse conceito é uma cadeia de dois manipuladores conectados por um ponto de
apoio capaz de fixar-se à pá. O apoio serve como forma reduzir o torque
necessário nas juntas ao reduzir a alcance necessário para cada manipulador
individualmente.

Para facilitar futuras referências nesse texto, o braço robótico que se
apoia sobre o chão será chamado de primário, manipulador que parte dele de
secundário e o ponto de apoio entre eles, capaz de fixa-se à pá, será chamado de
fixador.

A pistola de metalização presa ao manipulador secundário deve ter
alcance sobre toda a pá da turbina. Para isso existe uma gama de pontos
necessários onde o fixador deve ser posicionado. Com esse intuito o
manipulador primário da cadeia precisa ser projetado para ter uma região de
trabalho com um alcance total sobre esses pontos. Para tal, informações precisas
sobre o formato das pás são necessárias.

Para viabilizar a solução, é necessário verificar quais são as soluções
possíveis para gerar a aderencia do fixador sobre à pá. As tecnologias mais
difundidas são por força magnética e por diferença de pressão (ventosa).
As maiores preocupações com o relação a adesão são a capacidade de carga do
método e a resistência do \textit{coating}. As soluções por magnetismo e
por ventosa afetam de maneira diferente as camadas do material ao qual se
aderem. Enquanto o magnetismo atrai ativamente o material ao qual prentende
aderir, a ventosa apenas reduz a pressão do ar na região onde ela se fixa. Ambas
as soluções possuem versões ativas e passivas, assim como uma gama de opções com
relação à capacidade de suportar carga. Logo, para essas soluções, a carga
necessária a ser suportada, o magnetismo do material, a resistência à baixa
pressão do \textit{coating} e os efeitos da atração magnética, também, sobre o
\textit{coating} são perguntas que devem ser respondidas.

O braço secundário deve ser projetado para atingir os requerimentos de
posicionamento e velocidade da pistola de metalização, porém a região sob o
fixador, certamente, não estará disponivel para receber o revestimento. Assim, a
possibilidade, ou os requisitos necessários, de mover o fixador para uma região
da pá recém metalizada deve ser vista analisada antes do robô bipartido ser
considerado uma possibilidade.

\subsection{Robô Pendurado} 
